% !TEX root = thesis.tex
% Guillaume Fraux PhD thesis -- (C) 2019
% Distributed under CC-BY-SA-NC license 4.0
% Creative Commons Attribution-NonCommercial-ShareAlike 4.0 International

\chapter*{General introduction}
\pagenumbering{arabic}

Nanoporous materials are material presenting nanometric cavities in their
structure. These cavities (called \emph{pores}) and the associated high specific
surface area are the foundation for crucial industrial applications, in
particular in the domains of fluid separation and storage, water purification,
and heterogeneous catalysis. Two classes of crystalline nanoporous materials
that particularly stand out today are zeolites and metal--organic frameworks.
Zeolites are natural and artificial porous aluminosilicate known since 1756 and
artificially synthesized since the 1940s. They are currently highly used at the
industrial level, in diverse applications including as catalysts in the oil
industry and water softeners in laundry detergents. Since the 2000s, a new
family of hybrid organic--inorganic crystalline nanoporous materials called
metal--organic frameworks have emerged and sparked interest in the scientific
community. These new materials are based on metallic clusters, linked together
by organic linkers. This composition offers them immense structural and chemical
diversity makes them tuneable: it is possible to engineer new materials with
specific pore sizes, shapes, and chemistry by using different combinations of
linkers and metals. The relatively weak coordination bonds between the metals
cations and organic linkers create intrinsic structural flexibility in
metal--organic frameworks, which can be either local or extended to the whole
material. Some of these metal--organic frameworks --- grouped together under the
name \emph{"soft porous crystal"} --- exhibit large scale structural
transformations under external stimuli such as temperature, pressure, adsorption
or even exposure to light.

Most if not all applications of nanoporous materials are related to the entry of
other chemical species (from the liquid or gas phase) inside the material's
pores. When the external fluid is in the gaseous state, the process is called
adsorption, while for liquids it is called intrusion. Both adsorption and
intrusion have an effect on the physical and chemical properties of the fluid
confined in the nanopores. Confined fluids are usually organized more regularly,
taking some aspect of a solid phase while remaining mobile. The opposite is also
true: the presence of the fluid inside the pores can modify the properties and
behavior of the surrounding material. Flexible materials are especially impacted
and can undergo adsorption-induced phase transitions, resulting in macroscopic
phenomena like gate opening, breathing or negative gas adsorption. The coupling
between adsorption or intrusion and changes in the structure of flexible
nanoporous materials is difficult to study, because it involves the equilibrium
between confined and bulk fluids, as well as equilibrium between different
phases of the material.

\newpage
I have been interested during my PhD in the molecular simulation of fluid
adsorption and intrusion in flexible nanoporous materials. Molecular simulation
tools can speed up the development of new materials tailored to specific
applications by predicting the properties of materials before they are
synthesized, lowering the cost of screening thousands of materials. Which
property can be studied, and the reliability of the corresponding prediction
depend on the techniques used to model the systems of interest. Over the course
of my PhD, I used multiple techniques at different time and length scales to
look at adsorption and intrusion in flexible porous materials, and the effects
on both the confined fluid and the materials. I used macroscopic models based on
classical thermodynamics for the study of co-adsorption of gases, umbrella
sampling simulations to extract free energy profiles during water intrusion,
classical molecular dynamics and Monte Carlo for the intrusion of water and
aqueous solutions in both rigid and flexible materials, and \abinitio molecular
dynamics simulations to investigate the molecular organization of gases inside
pores.

Unfortunately, current molecular simulations methods only allow looking at the
coupling between adsorption and deformations from a single point of view.
Molecular dynamics simulations can be used to describe deformations of a
supramolecular framework, but are unable to simulate open systems, in particular
the adsorption of particles in chemical equilibrium with a reservoir. Monte
Carlo simulation, on the other hand, can describe such open systems and thus the
adsorption phenomenon, but are very inefficient for the study of collective
deformations. I have thus taken interest in (and worked on) the Hybrid Monte
Carlo simulation method, which can give us the best of both Monte Carlo and
molecular dynamics simulations. I also approached molecular simulation in
general and Hybrid Monte Carlo in particular from the point of view of their
implementation in molecular simulation software, studying various simulations
techniques, their links to statistical thermodynamics and efficient
implementation strategy.

\begin{center}
    \pgfornament[width=6cm,color=CTsemi]{88}
\end{center}

This thesis presents my work on the molecular simulation of adsorption and
intrusion in flexible nanoporous materials, and is divided into six chapters. I
start by presenting nanoporous materials, in particular zeolites, metal--organic
frameworks and zeolitic imidazolate frameworks. I introduce the main
characteristics at the origin of their huge structural diversity. I also
describe the phenomena of adsorption and intrusion, how they relate to one
another, and what effects they can have both on nanoporous solids and on
confined fluids. In a second chapter, I review the classical thermodynamic
models of adsorption and co-adsorption, and then demonstrate how to use
classical thermodynamics with the OFAST theory to study co-adsorption in
flexible nanoporous materials. My third chapter establishes statistical
thermodynamics and the molecular simulations methods we can use to study
chemical systems: molecular dynamics and Metropolis Monte Carlo. I also briefly
discuss different techniques used to model the interactions between molecules,
such as \abinitio methods and classical force-fields. In chapter four, I present
the basis for density functional theory calculations, which I have used to study
the adsorption of nitrogen in \ZIF8, showing that the confined fluid undergoes a
reorganization linked to framework changes as the pressure increases. The fifth
chapter contains classical molecular dynamics studies of the adsorption and
intrusion of water in nanoporous solids. I studied intrusion of electrolyte
aqueous solutions in \ZIF8, and its impact on the fluid structure, dynamics, and
thermodynamics. I also looked at the confinement of water in aluminosilicate
nanotubes, and the corresponding impacts on structure and dynamics. The last
chapter describes my work on the implementation of molecular simulation
software, and in particular the Hybrid Monte Carlo simulation method, which is
particularly suited to the study of adsorption in flexible nanoporous materials.
Finally, I present some conclusion from my PhD project, and identify some
perspectives and challenges for future work.

\vfill
\begin{center}
    \pgfornament[width=6cm,color=CTsemi]{75}
\end{center}
\vfill\vfill
