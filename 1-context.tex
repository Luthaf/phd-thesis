\documentclass[thesis]{subfiles}

\begin{document}

\chapter{Context of this work}

\section{Flexible nanoporous materials}

% Among the porous materials used commercially, one can list
% inorganic materials (such as zeolites and silica gels), carbon-based compounds
% (e.g., activated carbon), and hybrid organic--inorganic materials, including the
% topical family of metal-organic frameworks or porous coordination polymers.

% \subsection{ZIF}
%
% ZIFs are built around tetravalent metal centers such as Fe, Co, Zn, Cd or Cu;
% linked together by imidazolate linkers.  They also present the same topology as
% zeolites, with metal(imidazolate)$_2$ building blocks taking the role of
% \ce{SiO2}. The first ZIFs (ZIF-1 to ZIF-12) were synthesized in
% 2006\cite{Park2006}, and found to be water and thermally resistant, which made
% them interesting MOF for commercial applications.

% Nanoporous crystalline materials such as zeolites, metal--organic frameworks
% (MOF), carbon nanotubes and inorganic open frameworks enjoy a wide range of
% applications, ranging from catalysis, fluid separation and purification, to gas
% capture and detection of dangerous molecules. The fundamental basis for most of
% these applications is the material's capability to adsorb large quantities of
% molecules inside its nanometer-sized pores, due to its large specific surface
% area.\cite{Rouquerol2013} In this context, hydrophobic nanoporous
% materials offer the advantage that the uptake of water from the gas phase ---
% e.g., humidity in the air --- is very small.\cite{Wu2010, Ghosh2013, Wang2016}
% Given that water is often strongly adsorbed and competes with other molecules
% for adsorption sites, hydrophobic molecular sieves can offer higher separation
% properties.\cite{Flanigen1978, Giaya2000}

% \ZIF8 is a nanoporous material part of the Zeolitic Imidazolate
% Frameworks (ZIF) family. These MOF are built from imidazolate anions as their
% organic linker (mim = 2-methylimidazolate in the case of \ZIF8) and a divalent
% metal cation (here, \ce{Zn^2+}). The frameworks of ZIFs are four-connected,
% meaning that the ZIFs adopt zeolitic topologies, with the metal center replacing
% the tetrahedral silicon atom and the linker replacing the oxygen atoms in the
% \ce{SiO2} building block. \ZIF8 has formula \ce{Zn(mim)2} and adopts the
% sodalite (\textbf{sod}) topology. In this topology, large quasi-spherical pores
% corresponding to the sodalite cages are connected by windows formed by 6 and 4
% zinc atoms (see figure~\ref{fig:SOD}). In \ZIF8, the 4 members windows are too
% small for any molecules to go through, and all of the connectivity of the pore
% space happens though the 6 members windows.

\section{Adsorption and Intrusion}

\section{Molecular simulations}

\subsection{Atoms and molecules}

\subsection{System size and periodic boundary conditions}

\OnlyInSubfile{\printbibliography}

\end{document}
