% !TEX root = thesis.tex

\chapter*{General conclusions}

The work presented in this thesis is related to the study of adsorption and
intrusion in nanoporous flexible materials, the deformation of these materials
and the coupling between the two phenomena. Confining a fluid inside a porous
network has significant effects on its thermodynamics properties, due to the
competition between pore size and pore shape effects, and interface
interactions. This competition generates specific new behaviors, such as new
fluid phases and phases transitions, and is especially acute in nanoporous
materials, where the typical width of the pore and the range of the interactions
are of the same order of magnitude. On the other hand, the presence of a
confined fluid can also have strong effects on the surrounding solid, creating the
opportunity for new phases to exist and shifting the balance between multiple
meta-stable phases. This is particularly poignant in the case of \emph{flexible}
nanoporous material, such as many metal--organic frameworks.

Because these materials are relatively recent, their flexibility has often been
overlooked, and it was only in the recent years that the scientific community
started to take it into account. An example of such shift is presented in the
second chapter of this manuscript, with the incorporation of the osmotic
thermodynamic ensemble into the Ideal Adsorbed Solution Theory (IAST) for the
study of co-adsorption of gases, leading to the creation of the Osmotic
Framework Adsorbed Solution Theory (OFAST). In the aforementioned chapter, I
demonstrate that IAST is by construction invalid for the treatment of
co-adsorption when the adsorbing host is not inert during adsorption. In
particular, I show that IAST cannot be used for the prediction of co-adsorption
of fluid mixtures frameworks presenting a gate-opening behavior, and that it
predicts non-physical selectivity, up to two orders of magnitude higher than
OFAST. Even when IAST is not explicitly used to compute selectivity in flexible
frameworks, researchers should be careful when comparing single-component
isotherms of different guests in presence of flexibility. Differences in step
pressure of stepped isotherms can lead to claims of strong selectivity, when
applying concepts that are valid only for rigid host matrices.

We should also take care of not going too far the other way either, and
attributing all behaviors observed to the flexibility of the material. In the third
chapter of this document, I used \abinitio molecular dynamics simulations to
explain the origin of a stepped isotherm for the adsorption of nitrogen in
\ZIFCH3 and \ZIFCl, and its absence in the similar framework of \ZIFBr. I showed
that while the framework does deform during adsorption for both \ZIFCH3 and
\ZIFCl, the deformations do not change the accessible volume and pore size
distribution of these materials. Instead, the increase in uptake in the isotherm
is linked to a reorganization of the fluid confined in the pores, reorganization
which does not happen in \ZIFBr because of the difference in pore size. It is
thus fundamental to account for both flexibility and confinement effects when
studying adsorption in soft porous crystals.

The same is true of intrusion, adsorption's big brother. In chapter five, I used
classical molecular simulations to study the confinement under high-pressure of
water and aqueous electrolyte solution in \ZIF8, and imogolite nanotubes. I
observed confinement effects ranging from stronger spatial organization, changes
in elastic properties, to water dynamics slowdown. Interestingly, the presence
of ions at high concentration can have the same effects on bulk water,
structuring the hydrogen bonds network and slowing down dynamics. The intrusion
of aqueous solution in hydrophobic material is a promising way to store and
dissipate mechanical energy. Adding electrolyte in the water at varying
concentration allows tuning the behavior and even switching from energy storage
to dissipation. I looked at the impact of the ions on the intrusion behavior
using umbrella sampling simulations to extract the free energy profile of entry
inside \ZIF8, showing that different ions have dissimilar barriers when
traversing \ZIF8 windows. This study is one of the first on the subject of
intrusion of electrolytes in metal--organic frameworks, and allowed to shed some
light on the complex behaviors emerging in these systems.

This need to simultaneously account for adsorption and deformations was a
recurrent theme of all these studies. But current simulations methods are only
able to address one dimension of the problem: molecular dynamics simulations can
describe deformations, but modeling open systems and thus adsorption is not
possible. Metropolis Monte Carlo simulations can be used for open systems, but
have difficulties efficiently sampling collective deformations. Hybrid Monte
Carlo is a possible answer to this dilemma, combining the efficiency of
molecular dynamics with the versatility of Monte Carlo simulations (in
particular the possibility of sampling open and extended ensembles). My last
chapter presents the hybrid Monte Carlo simulation method and how to use it for
direct simulations in the osmotic ensemble. Increasing the reach of such novel
methods within the scientific community requires them to be readily available in
generic, easy to use, and efficient software. I explored the design space for
such software with the Domino project, described in the last chapter of this
document.

There is another condition to fulfill before being able to widely use osmotic
ensemble simulations for the study of adsorption and intrusion in soft porous
crystals. We need to be able to predict the energy changes related to the
flexibility of the frameworks and their interactions with fluids. On one hand,
first-principle or \abinitio methods (such as the density functional theory)
enable to accurately compute the energy of arbitrary atomistic systems. On the
other hand, they require large amount of computational power, which prevents
them from being routinely used on large systems. When faced with such large
systems --- large in the number of atoms, the timescale of processes, or for
high-throughput screening --- we therefore often turn to classical force fields
instead.

Classical force fields are either accurate, \ie reproducing well the actual
potential energy surface; or transferable, \ie usable with multiple different
systems. Current transferable force fields are not well suited to describe the
flexibility arising from coordination bonds, so we need to create new force
fields for these systems. Historically, the parametrization of new force fields
has been a rather long and tedious process. In recent years, new machine
learning-based techniques for the consistent and fast derivation of accurate
force fields have been devised. I presented one of these techniques in the
fourth chapter of this document, and used it to derive force fields for \ZIF8
and some of its derivatives from \abinitio data. Such automated methods are
especially crucial for the study of metal--organic frameworks given the huge
diversity of their structures. I hope that the availability of both accurate
force fields and hybrid Monte Carlo simulations capable software will make it
easier to use molecular simulations to engineer new materials tailored for
specific applications.

Overall, this PhD presents multiple molecular modeling methods that can be used
for the study of adsorption and intrusion in flexible nanoporous materials. From
\abinitio simulations to remove the need for force field parametrization;
classical simulations using molecular dynamics to describe flexibility; Monte
Carlo and particularly Grand Canonical Monte Carlo for adsorption; free energy
methods such a umbrella sampling; to hybrid Monte Carlo for the direct
simulation of collective behaviors in open systems; all the way to macroscopic
modeling methods, based on classical thermodynamics.

\begin{center}
    \pgfornament[width=6cm,color=CTsemi]{88}
\end{center}

This work opens perspectives for improvements in various directions. Starting
with molecular simulation methods, hybrid Monte Carlo seems like a very powerful
technique, that can be used for a wide variety of systems outside of the problem
of adsorption in soft porous crystals. First, hybrid Monte Carlo, being a
Metropolis Monte Carlo method, will always converge to and sample the phase
space distribution of the correct statistical thermodynamic ensemble. This is in
opposition with molecular dynamics, which samples the micro-canonical ensemble
by default, and relies on thermostats and barostats to sample other ensembles.
These thermostats and especially barostats are not all equal, and only some
algorithms are able to precisely generate the correct ensemble. At the same
time, hybrid moves greatly improve the efficiency of Monte Carlo simulations by
taking into account the local curvature of the potential energy surface.

Whether osmostats --- and the simulation of open systems --- are compatible with
molecular dynamics is still an open research question, while they are routinely
used in Grand Canonical and Gibbs ensemble Monte Carlo. Hybrid Monte Carlo could
thus be used for simulations of open ensembles and dilute systems, such as
constant pH simulations, description of the ionic environment of proteins, or
the simulation of defects in crystalline materials. Thanks to the Metropolis
Monte Carlo acceptance scheme, there is no need for the short molecular dynamics
run used inside hybrid move to sample any actual thermodynamic ensemble or any
physical Hamiltonian. This property could be exploited to create even more
efficient Monte Carlo simulations, for example gradually inserting new molecules
in a system while simultaneously relaxing its environment. Compressible
generalized hybrid Monte Carlo is particularly promising, as it enables using a
custom Hamiltonian tailored for the problem at hand.

Another perspective for future work concerns classical force fields, and their
ability to accurately reproduce potential energy surfaces. The traditional
approach to force fields has been to decompose the energy as a sum of terms
depending on simple scalar values with physical meaning: distances, bond length,
bond angles, torsion dihedral angles, \etc. These scalar values are then
combined with simple mathematical expressions, such as power or exponential
functions. This approach prevents the potential energy surface from including
multi-body effects, and accurately reproducing the shape of \abinitio potential
energy surfaces used as references. Machine learning tools, in particular neural
network and Gaussian processes, can improve both areas. First, neural-networks
ability to reproduce arbitrary functions from $\mathds{R}^n$ to $\mathds{R}$ can
reduce the differences between the potential energy surfaces shapes. For
example, instead of imposing the Lennard-Jones functional form, neural networks
can reproduce the actual function. Secondly, machine learning algorithms can be
coupled with better descriptors of the atomic structure, accounting for
many-body effects. In recent years, multiple independent scientific teams worked
to design, train and evaluate such machine learning force fields and the
associated descriptors. To my knowledge, they remain to be used for the
simulation of nanoporous flexible materials.

% \vfill
% \begin{center}
%     \pgfornament[width=6cm,color=CTsemi]{75}
% \end{center}
% \vfill\vfill
