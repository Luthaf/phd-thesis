% !TEX root = thesis.tex

\chapter*{General conclusions}

The works presented in this thesis are related to the study of adsorption and
intrusion in nanoporous flexible materials, deformation of these materials and
the coupling between the two phenomenon. Confining a fluid inside a pores
network has significant effects on its thermodynamics property, due to the
competition between pore size and pore shape effects, and interface
interactions. This competition generates specific new behaviors, such as new
fluid phases and phases transitions, and is especially acute in nanoporous
materials, where the typical of the pore and the range of the interactions are
of the same order of magnitude. On the other hand, the presence of a confined
fluid can also have strong effects on the solid, creating the opportunity for
new phases to exists and shifting the balance between multiple meta-stable
phases. This is particularly poignant in the case of \emph{flexible} nanoporous
material, such as metal--organic frameworks.

Because these materials are relatively recent, their flexibility have often been
overlooked, and it was only in the recent years that the scientific community
stared to take it into account. An example of such shift is presented in the
second chapter of this manuscript, with the incorporation of the osmotic
thermodynamic ensemble into the Ideal Adsorbed Solution Theory (IAST) for the
study of co-adsorption of gases, leading to the creation of the Osmotic
Framework Adsorbed Solution Theory (OFAST). In the aforementioned chapter, I
demonstrate that IAST is by construction invalid for the treatment of
co-adsorption when the adsorbing host is not inert during adsorption. In
particular, I show that IAST can not be used for the prediction of co-adsorption
of fluid mixtures frameworks presenting a gate-opening behavior, and that it
predicts non-physical selectivity, up to two orders of magnitude higher than
OFAST. Even when IAST is not explicitly used to compute selectivity in flexible
framework, researchers should be cautious cautious when comparing
single-component isotherms of different guests in presence of flexibility.
Differences in step pressure of stepped isotherms can lead to claims of strong
selectivity, when applying concepts that are valid only for rigid host matrices.

We should also take care of not going too far the other way either, and
attributing all behaviors to the flexibility of the material. In the third
chapter of this document, I used \abinitio molecular dynamics simulation to
explain the origin of a stepped isotherm for the adsorption of nitrogen in
\ZIFCH3 and \ZIFCl, and its absence in the similar framework of \ZIFBr. I showed
that while the framework does deform during adsorption for both \ZIFCH3 and
\ZIFCl, the deformations do not change the accessible volume and pore size
distribution of these materials. Instead, the increase in uptake in the isotherm
is linked to a reorganization of the fluid confined in the pores, reorganization
which does not happen in \ZIFBr because of the difference in pore size. It is
thus fundamental to account for both flexibility and confinement effects when
studying adsorption in soft porous crystals.

The same is true of intrusion, adsorption's big brother. In my chapter five, I
used classical molecular simulations to study the entry of water and aqueous
electrolyte solution in \ZIF8, and imogolite nanotubes. I was able to show some
of the effects confinement can have on water, such as a stronger spatial
organization, and a slower dynamics. Interestingly, the presence of ions at high
concentration can have the same effects on bulk water, structuring the hydrogen
bonds network and slowing down dynamics. I could also shed some light on the
impact of the ions on the overall intrusion behavior, showing that different
ions have different barriers to entry through \ZIF8 windows. \fbox{TODO}


=> Study both adsorption/intrusion and deformations at the same time!

    => need new simulation methods, available in generic, easy to use, efficient software

=> Force field parametrization with ML


Overall, this PhD presents multiple molecular modeling methods that can be used
for the study of adsorption and intrusion in flexible nanoporous materials: from
\abinitio simulations to remove the need for force field parametrization, to
classical simulations using molecular dynamics or Monte Carlo, to hybrid Monte
Carlo for the joint simulation of open systems and collective behaviors, all the
way to macroscopic methods, using classical thermodynamics.

\newpage
\begin{center}
    \pgfornament[width=6cm,color=CTsemi]{88}
\end{center}
Perspectives:

    => opinion perso sur les challenges à venir

    => Advanced HMC (compressible, with bogus MD)

    => force fields, machine learning, NN potentials
