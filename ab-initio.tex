\documentclass[thesis]{subfiles}

\begin{document}

\OnlyInSubfile{\setcounter{chapter}{3}}

\chapter{Quantum simulations}
\startcontents[chapters]
\printpartialtoc

\section{From Schrödinger equation to Density Functional Theory}

\subsection{Quantum chemistry and Schrödinger equation}

If we want to be able to compute properties of an atomic system using the
framework of statistical physics, we need to be able to compute the total energy
of the system in a given configuration $U(r^N)$. The most generic way to do so
is directly derived from the fundamental equation of quantum physics, the
Schrödinger equation. In the quantum chemistry, the state of a system is
represented by a complex-valued function of positions and time $\Psi(r, t)$. The
Schrödinger equation describes the time evolution of the system given its
Hamiltonian $\hat{\mathcal{H}}$:
\[\hat{\mathcal{H}} \Psi(r, t) = i\hbar \frac{\partial}{\partial t} \Psi(r, t),\label{eq:time-schrodinger}\]
where $i$ is the imaginary number, and $\hbar$ the reduced Plank's constant. The
quantum Hamiltonian operator can be expressed as a sum of quantum kinetic energy
and potential energy:
\[\hat{\mathcal{H}} \Psi(r, t) = - \sum \frac{\hbar^2}{2m} \nabla^2 \Psi(r, t) + V(r^N)\Psi(r, t)\]
where the $m$ are the masses of the particles in the system, $\nabla$ is the
nabla operator, and $V(r)$ is the potential energy of the system. When the
potential energy does not depend on time, we can look for solution with
separated variables of the form $\Psi(r, t) = \psi(r) \times y(t)$. Replacing in
equation~\eqref{eq:time-schrodinger}, and separating the spatial and time
quantities, we get
\[- \sum \frac{\hbar^2}{2m} \frac{1}{\psi(r)} \nabla^2 \psi(r) + V(r^N) = i\hbar \frac{1}{y(t)} \frac{\d y(t)}{\d t}.\]
For this equation to stand for all $t$ and all $r$, there must exist a constant
value $E$ such that:
\[\begin{dcases}
    E =& - \sum \frac{\hbar^2}{2m} \frac{1}{\psi(r)} \nabla^2 \psi(r) + V(r^N) \\
    E =& i\hbar \frac{1}{y(t)} \frac{\d y(t)}{\d t} \\
\end{dcases}\]

Putting everything together, any solution of the general
equation~\eqref{eq:time-schrodinger} can be written as a (potential infinite)
linear combination of special solutions:
\[\Psi(r, t) = \sum_i c_i \ \psi_i(r) \ e^{-i E_i t / \hbar},\]
with $c_i$ being the complex coefficients of the combination; and the $(E_i\ ;\
\psi_i)$ pairs are solutions of the \emph{time-independent} Schrödinger equation:
\[\hat{\mathcal{H}} \psi(r) = E \psi(r).\label{eq:schrodinger}\]
Because the time dependence of $\Psi$ have a wave-like form and because $\Psi$
propagate through space and time, it is often called a \emph{wave function}.

\subsection{Density Functional Theory}

To compute the energy of an atomic system, we need to solve
equation~\eqref{eq:schrodinger} for a collections of $N$ electrons carrying the
negative charge $-e$ and $M$ nuclei carrying the positive charge $e Z_j$
interacting though Coulombic potential. As the electrons are much more
lightweight than the nuclei (an electron is 2000 time lighter than a single
proton), it is customary to work under the Born-Oppenheimer approximation. In
this approximation, the degrees of freedom of electrons and nuclei are
decoupled, and the electrons move much faster than the nuclei. This effectively
means that at a given point in time, the electrons evolve in a constant
electrostatic potential from the fixed nuclei. The corresponding Hamiltonian
(using atomic units, \ie $e^2 / 4\pi\epsilon_0 = 1$; $\hbar = 1$; and
$m_\text{electron} = 1$) is:
\[H\psi = \left[-\frac 1 2 \sum_i^N \nabla^2_i - \sum_i^N \sum_j^M \frac{Z_j}{|\r_i - \r_j|} + \sum_i^N \sum_{j>i}^N \frac{1}{|\r_i - \r_j|} \right] \psi\]
We can rewrite this by defining the potential an electron feels due to the
presence of all the nuclei $V_\text{ext}(r) = -\sum_{j} Z_j / |\r - \r_j|$, and
the electron-electron interaction potential $U(r_i, r_j) = 1 / |\r_i -
\r_j|$:
\[\hat{\mathcal{H}}\psi = \left[-\frac 12\sum_{i=1}^N \nabla^2_i + \sum_{i=1}^N V_\text{ext}(\r_i) + \sum_{i=1}^N \sum_{j>i}^N U(\r_i, \r_j)\right] \psi\]
\[\hat{\mathcal{H}}\psi = \left[\hat T + \hat V_\text{ext} + \hat V_{ee}\right] \psi \label{eq:electronic-hamiltonian}\]
The three terms in the Hamiltonian are the kinetic energy $\hat T$, the total
external potential $\hat V_\text{ext}$, and the electron-electron interaction
potential $\hat V_{ee}$.

The Density Functional Theory (DFT) is a strategy to solve the Schrödinger
equation without having to explicitly determine the wave function $\psi$. It is
based on the fact that the square of the norm of the wave function is the
probability for the system to be in a given state. For a system of electrons
evolving in the fixed potential created by the nuclei, the wave function
$\psi(\{\vec x\}, \{s\})$ depends on the Cartesian coordinates of all the
electrons $\{\vec x\}$ and their spins $\{s\}$. The electronic density $n(\vec
r)$ is the probability to find an electron in small neighboring of $\r$. It
is defined as:
\[n(\r) = N \iiint |\psi(\r, s_1, \vec x_2, s2, \cdots, \vec x_N, s_N)|^2 \ \d \vec x_2 \cdots\d \vec x_N \ \d s_1 \cdots\d s_N \label{eq:electron-density}\]
We also define the two electrons density $n_2(\r_1, \r_2)$, which is the
probability for an electron to be in $\r_1$, while another electron is in
$\r_2$:
\[n_2(\r_1, \r_2) = N \iiint |\psi(\r_1, s_1, \r_2, s_2, \vec x_3, s3, \cdots, \vec x_N, s_N)|^2 \ \d \vec x_3 \cdots\d \vec x_N \ \d s_1 \cdots\d s_N\]

Hohenberg and Kohn showed in 1964 that every electronic density correspond
exactly to one and only one external potential $V_\text{ext}$, \ie the knowledge
of the external potential or the density are equivalent. This also means that
knowing the electronic density is equivalent to knowing the wave function of the
system, as one can reconstruct it using equation~\eqref{eq:schrodinger}. As a
consequence, all the observables of the system only depend on the electronic
density, and can be written as functionals of this density: $E[n]$, $\Psi[n]$,
\etc

The second Hohenberg and Kohn theorem states that the ground state electronic
density $n_0(\r)$ is the one that minimize the energy functional: $E[n] \geq
E_0 = E[n_0]$. From the previous relations, the total energy functional contains
three terms:
\[E[n] = T[n] + V_\text{ext}[n] + V_{ee}[n];\]
where the potential terms can be expressed as integrals of the one and two
electrons density:
\[ V_\text{ext}[n] = \int V_\text{ext}(\r) \ n(\r) \ \d \r;\]
\[ V_{ee}[n] = \iint \frac{n_2(\r_1, \r_2)}{|\r_1 - \r_2|} \ \d \r_1 \d \r_2.\]
While we can compute and minimize $V_\text{ext}[n]$ given the positions of the
nuclei, the two other terms are harder to evaluate.

Kohn and Sham reformulated the problem in 1965 by considering a set of
\emph{non-interacting} electrons evolving in a specific external potential, such
as the density arising from these electrons is the same as the one we look for.
The new system of non-interacting electrons is described by a new set of
independent orbitals $\phi_i$, such as the total electronic density of the system
\[n(\r) = \sum_i^N \left| \phi_i(\r) \right|^2\]

For these non interacting electrons, the kinetic energy $T_s[n]$ is known:
\[T_s[n] = \sum_i^N \int \phi_i^*(\r) \left( -\frac 12 \nabla^2 \right) \phi_i(\r) \d \r\]
We also know that most of the electron-electron interactions will arise from the
classical Coulombic interaction --- the Hartree energy --- which we can compute
from the electronic density:
\[V_H[n] = \iint \frac{n(\r_1) n(\r_2)}{|\r_1 - \r_2|} \ \d \r_1 \d \r_2.\]
The functional to minimize in order to find the electronic density is now
\[E[n] = T_s[n] + V_\text{ext}[n] + V_H[n] + E_{xc}[n],\label{eq:energy-functional}\]
in which we can compute all of the terms expect for the exchange-correlation
contribution $E_{xc}[n]$.
\[E_{xc}[n] = (T[n] - T_s[n]) + (V_{ee}[n] - V_H[n]).\]
This contribution is generally unknown, and describe the quantum nature of
electrons that are able to interact with themselves.\fbox{interpretation of
exchange and correlation?} It can be shown that the minimizing the energy
functional is equivalent to solving a set of differential equations:
\[ \left[-\frac 12 \nabla^2 + V_\text{ext}(\r) + \frac{\partial V_H[n](\r)}{\partial n(\vec r)} + \frac{\partial E_{xc}[n](\r)}{\partial n(\vec r)} \right] \phi_i = \epsilon_i \phi_i \label{eq:kohn-sham}\]
These are non-linear equations, as both $V_H[n]$ and $E_{xc}[n]$ depends on the
$\phi_i$ through $n(\r)$. The usual algorithm to solve them is a self-consistent
iterative algorithm. Starting from an initial guess for the Kohn-Sham orbitals
$\phi_i^0$:
\begin{enumerate}
    \item Calculate the corresponding electron density at step $\alpha$, $n^\alpha(\r)$;
    \item Calculate all the terms in the left hand side of equation~\eqref{eq:kohn-sham}:
          kinetic, external, Hartree and exchange-correlation energies using this electron density;
    \item Solve the Kohn-Sham equation to find new Kohn-Sham orbitals $\phi_i^{\alpha+1}$;
    \item Compute the new energy using the energy functional~\eqref{eq:energy-functional}.
          If this energy is the same as the one in the previous step, then we have found
          the minimal energy and thus the ground state density. Else, go back to step
          1 using the new Kohn-Sham orbitals as the initial guess.
\end{enumerate}

\subsection{Exchange, Correlation and dispersion}

A good approximation for the exchange-correlation functional is required to
solve the Kohn-Sham equations. There are number of approach routinely used in
theoretical chemistry such as Local Density Approximation (LDA) where the
exchange-correlation functional only depends on the local density; Generalized
Gradient Approximation (GGA), where the functional also depends on the local
gradient of the density, meta-GGA functional incorporating the second derivative
of the density (the Laplacian of the density); or hybrid functional. Hybrid
functionals mix exchange expressions from LDA or GGA with the more precise
Hartree-Fock \fbox{introduce HF?} expressions, while using the LDA or GGA
expressions for the correlation.

Generally, the exchange-correlation functional only depends on the local
density, and maybe some of its derivative. This means that the DFT method will
have trouble reproducing the non-electrostatic long-range correlation effects
such as the dispersion interaction. Fortunately, we have a simple analytic
formulation for these interaction, which Grimme and coworkers\cite{Grimme2006}
proposed to include when computing the total energy. This correction is added to
the energy obtained by DFT, and does not directly modify the electronic density.
\[E_\text{tot} = E_\text{DFT} + E_\text{disp}; \]
This correction is added to the energy obtained by DFT, and does not directly
modify the electronic density.
\[\text{with}\quad E_\text{disp} = - s_6 \sum_i^M\sum_{j>i}^M\frac{\sqrt{C_6^i C_6^j}}{R_{ij}^6} f_{ij}(R_{ij}),\]
where the sum is over all the pairs of atoms, $R_{ij}$ is the distance between
the atoms, $s_6$ a parameter specifically adapted to the exchange-correlation
functional one uses, the $C_6^i$ are parameters tabulated for each element and
$f_{ij}$ is a damping function.
\[f_{ij}(R_{ij}) =
\begin{dcases}
    \frac{1}{1 + \exp\left[-d \left(\frac{R_{ij}}{\sigma_i + \sigma_j} -1\right)\right]} &\text{for } R_{ij} < R_c \\
    0 &\text{for } R_{ij} \geq R_c \\
\end{dcases}\]
In this expression, $d$ is a damping parameter, usually set to 20, $R_c$ a
cutoff radius, and the $\sigma_i$ parameters are tabulated atomic \vdW radii.

\section{Adsorption of \ce{N2} in \ZIF8 and derivatives}

\subsection{The \ZIF8 family of materials}

\subsection{Deformation of \ZIF8 under adsorption}

\subsection{Changes in the adsorbed phase}

\OnlyInSubfile{\printbibliography}

\end{document}
