\documentclass[thesis]{subfiles}

\begin{document}

\chapter{Context}
\startcontents[chapters]
\printpartialtoc

\section{Flexible Nanoporous materials}

\subsection{ZIF}

ZIFs are built around tetravalent metal centers such as Fe, Co, Zn, Cd or Cu;
linked together by imidazolate linkers.  They also present the same topology as
zeolites, with metal(imidazolate)$_2$ building blocks taking the role of
\ce{SiO2}. The first ZIFs (ZIF-1 to ZIF-12) were synthesized in
2006\cite{Park2006}, and found to be water and thermally resistant, which made
them interesting MOF for commercial applications.

\section{Adsorption and Intrusion}

\section{Molecular simulations}

\subsection{Computers and molecules}

\subsection{Atomistic view of molecules: quantum or classical}

\subsection{System size and periodic boundary conditions}

\OnlyInSubfile{\printbibliography}

\end{document}
