% !TEX root = thesis.tex
% Guillaume Fraux PhD thesis -- (C) 2019
% Distributed under CC-BY-SA-NC license 4.0
% Creative Commons Attribution-NonCommercial-ShareAlike 4.0 International

% information for cover pages

\institute{Chimie ParisTech}
\doctoralschool{Chimie Physique et\\ Chimie Analytique de\\ Paris Centre}{388}
\specialty{Chimie Physique}
\date{25 juin 2019}

\jurymember{1}{Sofía \textsc{Calero}}{Professeure, Universidad Pablo de Olavide}{Rapportrice}
\jurymember{2}{Paul \textsc{Fleurat-Lessard}}{Professeur, Université de Bourgogne}{Rapporteur}
% TODO: Président?
\jurymember{3}{Caroline \textsc{Mellot-Draznieks}}{Directrice de Recherche, Collège de France}{Examinatrice}
\jurymember{4}{Renaud \textsc{Denoyel}}{Directeur de Recherche, Aix-Marseille Université}{Examinateur}
\jurymember{5}{Alain \textsc{Fuchs}}{Professeur, PSL Université}{Examinateur}
\jurymember{6}{François-Xavier \textsc{Coudert}}{Chargé de Recherche, Chimie ParisTech}{Directeur de thèse}

\frabstract{

Durant ma thèse, j'ai utilisé la simulation moléculaire pour étudier
l'adsorption et l'intrusion de fluides dans les matériaux nanoporeux flexibles.
Aujourd'hui, les matériaux à charpente organo-métallique appelés
\emph{metal-organic frameworks} (MOF) sont les principaux représentants de cette
famille de matériaux. Je me suis en particulier intéressé à la \ZIF8, un MOF
constitué de zinc et de ligands imidazolates organisés dans une topologie de
type sodalite. Grâce à la dynamique moléculaire quantique, j'ai pu montrer que
lors de l'adsorption de diazote dans la \ZIF8 il se produit une réorganisation
de la phase adsorbée qui augmente la quantité totale d'azote adsorbée. J'ai
aussi montré que changer la nature chimique des ligands permettait de supprimer
partiellement ou totalement cette réorganisation.

D'autre part, j'ai utilisé la dynamique moléculaire classique et les simulations
Monte-Carlo pour étudier l'adsorption et l'intrusion d'eau dans des matériaux
poreux hydrophobes. Ces matériaux ont des applications potentielles dans le
domaine du stockage et de la dissipation de l'énergie mécanique. La pression à
laquelle se produit l'intrusion, ainsi que la présence et la forme d'une boucle
d'hystérèse sont modifiable par l'ajout d'ions dans le liquide d'intrusion.
J'ai montré que liquide confiné dans la \ZIF8 ou dans des nanotubes
d'alumino-silicates appelés imogolites est fortement structuré, et que la
dynamique des molécules d'eau est ralentie par le confinement. La présence
d'ions modifie très peu la structuration, mais ralenti encore la dynamique, et
rigidifie l'ensemble du système. J'ai aussi étudié l'entrée d'ions dans la
\ZIF8, et observé une différence flagrante entre \ce{Li+} et \ce{Cl-} d'un
point de vue thermodynamique et cinétique.

Enfin, j'ai montré que la prise en compte de la flexibilité était nécessaire
pour prédire correctement la co-adsorption de gaz dans un matériau qui se
déforme (\emph{respiration, ouverture des fenêtres}, \etc) lors de l'adsorption.
Cette prise en compte est possible dans le cadre de la méthode \emph{Osmotic
Framework Adsorbed Solution Theory} (OFAST) pour décrire des changements de
phase du matériau.

}

\enabstract{

During my PhD, I used molecular simulation to study the adsorption and intrusion
of fluids in flexible nanoporous materials. As of today, metal-organic
frameworks (MOF) are the main example of this family of materials. I
specifically worked with \ZIF8, a MOF built with zinc metallic centers and
imidazolates linkers, organized in a sodalite topology. Using \abinitio
molecular dynamics I showed that nitrogen undergoes a reorganization inside the
pores during adsorption; increasing the total adsorbed amount. I also showed
that changes to the chemical nature of linkers allows to partially or completely
remove this reorganization.

On an other side, I used classical molecular dynamics and Monte Carlo
simulations to study adsorption and intrusion of water in hydrophobic porous
materials. These materials have possible applications in mechanical energy
storage and dissipation. The intrusion pressure, as well as the presence and
shape of an hysteresis loop, can be tuned by adding ions in the intrusion
liquid. I showed that the liquid confined in \ZIF8 or in alumino-silicate
nanotubes called imogolites is strongly structured; and that the water molecules
dynamic slows down under confinement. The presence of ions almost does not
modify the water structuration, but slows down dynamics even more, and makes the
whole system more rigid. I also studied ions entry in \ZIF8 structure, and
observed a clear difference between \ce{Li+} and \ce{Cl-} both on a
thermodynamic and kinetic point of view.

Finally, I showed that it is necessary to take in account flexibility to
correctly predict gas co-adsorption in frameworks that undergoes deformation
(\emph{breathing, gate-opening,} \etc) under adsorption. This is possible within
the scope of Osmotic Framework Adsorbed Solution Theory (OFAST) for materials
undergoing phase transition.

}

\frkeywords{simulation moléculaire, matériaux nanoporeux flexibles, intrusion, adsorption, simulations multi-échelles}
\enkeywords{molecular simulation, flexible porous materials, intrusion, adsorption, multi-scale simulations}
