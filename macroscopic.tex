\documentclass[thesis]{subfiles}

\begin{document}

\OnlyInSubfile{\setcounter{chapter}{1}}

\chapter{Macroscopic studies} \label{sec:macroscopic}
\startcontents[chapters]
\printpartialtoc

\section{Thermodynamics, ensembles and potentials}

A first way to approach the problem of coupled deformation and adsorption is
through the use of macroscopic or thermodynamic methods. Such methods are based
on classical thermodynamic principles and allow us to gain high level
understanding of the processes at play. In this section, I will introduce the
concept of thermodynamic potential and thermodynamic ensemble that can be used
together to predict the evolution of a system. In a second time, I will show how
these potentials can be used to make predictions on the co-adsorption of
multiple gas in the same porous framework, but that one needs to be careful to
use the thermodynamic ensemble adapted to the problem at hand.

\subsection{The Laws of Thermodynamics}
\subsubsection{First law of thermodynamics}

The starting point for macroscopic methods are laws of thermodynamics. The first
law of thermodynamics is an generalization of the conservation of energy during
physical processes. It is  stated as follow:

\vskip0.8ex
\begin{quote}
    For a closed system, the change in energy between two states of internal
    thermodynamic equilibrium is the sum of the work of external forces and the
    heat received by the system.
\end{quote}
\vskip0.8ex

If we divide the total energy $E_\text{tot}$ of a system into its kinetic energy
$E_\text{kin}$, potential energy $E_\text{pot}$ and internal energy
$\mathcal{U}$, and letting $W$ be the work received by a system during a
transformation and $Q$ the heat transfer during the same transformation, the
first law of thermodynamics can be written as
\[\Delta E_\text{tot} = \Delta E_\text{kin} + \Delta E_\text{pot} + \Delta \mathcal{U} = W + Q.\]
If now we consider a system at rest in a constant external potential field,
$E_\text{kin}$ and $E_\text{pot}$ are constant, and we find the usual
formulation of the first law of thermodynamics:
\[\Delta \mathcal{U} = W + Q.\]
It is often interesting to use a differential formulation of this relation,
considering infinitesimal changes in $\mathcal{U}$, $W$, and $Q$:
\[\d \mathcal{U} = \delta Q + \delta W.\]
If the system composition is not constant, \ie if there are undergoing chemical
reactions or if the system is open and exchanges particles with an external
reservoir, we need to add another term depending on the quantity of mater for
each chemical species in the system $\{N_i\}$:
\[\d \mathcal{U} = \delta Q + \delta W - \sum_i \mu_i \d N_i.\label{eq:first-principle}\]
In this formulation, the $\mu_i$ are called the \emph{chemical potentials} of
the species. They represent the relative stability of the different chemical
species in the system.

An interesting special case is the one of an external pressures acting on the
system. In this case,
\[\delta W_\text{pressure} = -p_\text{ext} \d V \ ;\]
where $p_\text{ext}$ is the external pressure. If $p_\text{ext}$ is constant
during the transformation, this can be written as $\delta W_\text{pressure} =
-\d (p_\text{ext} V)$. We can then define the enthalpy $H$ as $H = \mathcal{U} +
PV$ ; which allow us to rewrite equation~\eqref{eq:first-principle} as
\[\d H = \delta Q + \delta W_\text{other} - \sum_i \mu_i \d N_i.\]
Here, $\delta W_\text{other}$ is the work coming from all forces
\emph{excluding} the forces of pressure.

\subsubsection{Second law of thermodynamics}

The second law of thermodynamics allow us to make predictions on the natural
evolution of a system.

\vskip0.8ex
\begin{quote}
    There exist a function of state called entropy, noted $S$, which is an
    increasing function of time for any transformation of a closed system.
\end{quote}
\vskip0.8ex

The variation of entropy during a transformation of a closed system can be
linked to the heat transfer between the system and its surroundings by the
Clausius relation:
\[\Delta S = \frac Q T + S_\text{created}. \label{eq:clausius}\]
The $Q/T$ term is homogeneous to an entropy and called the exchanged entropy,
and is the entropy given to the system by its surroundings. $S_\text{created}$
is the entropy created during the transformation, and is a positive quantity by
the second law of thermodynamics. When $S_\text{created}$ is null, the
transformation is said to be \emph{reversible}.

Using the relations~\eqref{eq:first-principle} and~\eqref{eq:clausius}; and
considering that only the pressure forces act on the system, we can extract an
evolution principle for any transformation:
\[T \d S = \delta Q + \d S_\text{created}\]
\[T \d S = (\d \mathcal{U} - \delta W - \sum_i \mu_i \d N_i) + \d S_\text{created}\]
\[\d \mathcal{U} + P \d V  - \sum_i \mu_i \d N_i - T \d S = -T\delta S_\text{created}.\label{eq:thermo-potentials}\]
If there is a function $\Phi$ such as $\d \Phi = \d \mathcal{U} + P \d V - \sum_i \mu_i
N_i - T \d S$, then for any transformation of a closed system, $\d \Phi =
-T\delta S_\text{created} \leq 0$. This means that $\Phi$ is a decreasing
function of time, and it will be minimal in the equilibrium state. In this case,
$\Phi$ is called the \emph{thermodynamic potential}.

\subsection{Thermodynamic ensembles}

A thermodynamic ensembles is a set of transformations of a system that obey some
constrains. For example the ensemble where the composition of the system is
fixed, together with the volume and total energy is called the $NVE$ or
micro-canonical ensemble. Other well known ensembles present a fixed pressure or
temperature: the $NVT$ or canonical ensemble, and the $NPT$ or
isobaric-isothermal ensemble.

Each of these ensembles have an associated thermodynamic potential. For the
micro-canonical ensemble, we start with equation~\eqref{eq:thermo-potentials}:
\[\d \Phi = \d \mathcal{U} + P \d V - \sum_i \mu_i \d N_i - T \d S.\]
$\d \mathcal{U}$ and $\d V$ will be zero as the energy and volume is constant.
Moreover as the system composition is fixed, $\d N_i = 0$. This gives
\[\d \Phi = - T \d S \leq 0.\]
The thermodynamic potential for the $NVE$ ensemble is the negentropy $N \equiv
-TS$, which will be minimal (and the entropy will be maximal) at equilibrium.

If we now allow the energy to change but fix the kinetic energy and thus the
temperature by the mean of an external thermostat,
equation~\eqref{eq:thermo-potentials} becomes
\[\d \Phi = \d \mathcal{U} + P \d V - \sum_i \mu_i \d N_i - T \d S - (S \d T - S \d T) \]
\[\d \Phi = \d \mathcal{U} + P \d V - \sum_i \mu_i \d N_i - \d (TS) + S \d T. \]
As the temperature, volume and composition are fixed, $\d T = \d V = d N_i = 0$,
which means that
\[\d \Phi = \d (\mathcal{U} - TS).\]
We can introduce the Helmholtz free energy $F \equiv \mathcal{U} - TS$
(sometimes simply called free energy), which is the thermodynamic potential in
the $NVT$ ensemble.

If now we allow the volume to change in the system, while fixing the pressure
with a barostat and keeping the temperature fixed,
equation~\eqref{eq:thermo-potentials} becomes
\[\d \Phi = \d \mathcal{U} + P \d V + (V\d P - V\d P) - T \d S - (S\d T - S\d T) - \sum_i \mu_i \d N_i\]
\[\d \Phi = \d \mathcal{U} + \d (PV) - \d(TS) - V\d P + S\d T - \sum_i \mu_i \d N_i\]
Again, as the pressure, temperature and composition are fixed, this reduces to
\[\d \Phi = \d \mathcal{U} + \d (PV) - \d(TS).\]
We define the Gibbs free energy $G$ as $G \equiv \mathcal{U} + PV - TS \equiv F + PV
\equiv H - TS$, which is the thermodynamic potential in the $NPT$ ensemble.

\subsection{Ensembles for adsorption processes}
\label{sec:osmotic-ensemble}

Getting back to the adsorption in porous material, we need to describe the
thermodynamic ensemble in which adsorption process takes place. The temperature
is always fixed, as well as the composition of the adsorbing host. If the host
is rigid, and do not deform under adsorption, then we can consider that the
volume of the system is constant. The main difference with the previously
defined ensembles is that the composition of the gas phase is not fixed:
the number of gas molecules adsorbed in the system is allowed to change during
adsorption. In this case, equation~\eqref{eq:thermo-potentials} becomes
\[\d \Phi = \d \mathcal{U} - \d(TS) + P\d V + S\d T - \sum_i \mu_i \d N_i\]
Removing the null terms $\d V$ and $\d T$, we find the definition of the
\emph{grand-canonical} thermodynamic potential $\Psi$, associated with the
$\mu_iVT$ ensemble:
\[\Psi \equiv \mathcal{U} - TS - \sum_i \mu_i N_i \equiv F - \sum_i \mu_i N_i.\]

If we now consider adsorption in a flexible host, the volume is no longer fixed.
Instead, in most experimental setups, external pressure is fixed and set equal
to the pressure of the adsorbed gases outside of the host. The thermodynamic
ensemble suited for the study of adsorption in flexible materials is called the
\emph{osmotic ensemble}, first introduced in 1994\cite{Mehta1994} for the study
of fluid mixtures, and adapted to multi-components phase equilibrium in
1998\cite{Escobedo1998}. Again, starting from equation~\eqref{eq:thermo-potentials}
\[\d \Phi = \d \mathcal{U} - \d (PV) - \d(TS) - V\d P + S\d T - \sum_i \mu_i \d N_i.\]
As usual, $\d T$ and $\d P$ are null as the corresponding variable is fixed. In
the sum over the $\mu_iN_i$ terms, only the adsorbed species quantities of
matter are allowed to change. The osmotic potential is thus defined by
\[\Omega \equiv \mathcal{U} - TS + PV - \sum_i \mu_i N_i \equiv F + PV - \sum_i \mu_i N_i \equiv G - \sum_i \mu_i N_i\]
This potential will be the basis we use to predict co-adsorption in flexible
porous media in the next section. The table~\ref{table:thermo-potential}
presents a summary of all the thermodynamic ensembles discussed so far, as well
as the associated fixed quantities and thermodynamic potentials.

\begin{table}[htp]
    \centering
    \renewcommand{\arraystretch}{1.3}
    \begin{tabularx}{0.8\textwidth}{l c c c X}
        Ensemble            & Fixed quantities                    & \multicolumn{3}{l}{Thermodynamic potential} \\ \hline
        Micro-canonical     & $N$, $V$, $E$                       & \hskip1em $N$      & $\equiv$ & $-TS$                   \\
        Canonical           & $N$, $V$, $T$                       & \hskip1em $F$      & $\equiv$ & $\mathcal{U} - TS$      \\
        Isobaric-Isothermal & $N$, $P$, $T$                       & \hskip1em $G$      & $\equiv$ & $H - TS$                \\
        Grand-Canonical     & $\mu_i$, $V$, $T$                   & \hskip1em $\Psi$   & $\equiv$ & $F - \sum_i \mu_i N_i $ \\
        Osmotic             & $N_\text{host}$, $\mu_i$, $P$, $T$  & \hskip1em $\Omega$ & $\equiv$ & $G - \sum_i \mu_i N_i $ \\
    \end{tabularx}
    \caption{Thermodynamic ensembles and associated thermodynamic potentials}
    \label{table:thermo-potential}
\end{table}

\section{Macroscopic calculations for gas separation in flexible materials}

Gas separation is an important step in multiple industrial processes, from
separation of hydrocarbons in oil chemistry to \ce{CO2} separation and storage
or oxygen extraction in the air. The two main methods used for gas separation
are cryogenic distillation, mainly used for air separation, and differential
adsorption. Adsorption-based processes for gas separation, which rely on
microporous materials as an adsorber bed, are very versatile because of the
large choice of materials available --- and the possibility to tune them for a
specific system.

Experimental characterization of the co-adsorption of a mixture of gases inside
a porous adsorbent is typically done through multi-component gas adsorption
studies. This problem is inherently high-dimensional, e.g., for a ternary
mixture there are four variables to vary (temperature, total pressure, and two
independent variables for the mixture composition). Because such experimental
studies of co-adsorption equilibrium thermodynamics are typically long and
expensive, there has been a great expense of literature devoted to theoretical
models for the prediction of mixture co-adsorption based on single-component
adsorption data. The most commonly used method in the field is the Ideal
Adsorbed Solution Theory (IAST)\cite{Myers1965}, which is relatively simple to
implement and robust, and allows the prediction of multi-component adsorption
behavior from individual single-component isotherms.

Materials which undergo large-scale reversible structural transitions impacting
their total volume or internal pore volume appear to be particular common among
metal-organic frameworks based on relatively weaker bonds (coordination bonds,
$\pi$--$\pi$ stacking, hydrogen bonds, or some covalent bonds) compared to
inorganic dense nanoporous materials (such as zeolites). In particular, some of
these materials show transitions between an \emph{open} phase with large pore
volume, and a \emph{condensed} or \emph{narrow pore} phase with smaller pore
volume --- or, in some cases, no microporosity at all. Such transitions, known
as \emph{gate opening}\cite{Kitaura2003, Tanaka2008, Li2015} or
\emph{breathing}\cite{Serre2002} depending on the order in which the phases
occur upon adsorption, can lead to stepped adsorption isotherms.

In the recent literature, many authors have relied on IAST predictions to
predict that several such flexible MOFs would present very good selectivity for
gas separation. In some cases, the authors explicitly used IAST to derive such
predictions on flexible materials\cite{Banerjee2014, Mukherjee2015, Foo2016,
Li2016}. In other cases, IAST was not used explicitly, but the assumptions made
for the behavior of mixtures stem from the \emph{classical} understanding of
selectivity rules in rigid materials, and would not necessarily be valid in
flexible materials\cite{Gucuyener2010, Inubushi2010, Nijem2012, Sanda2013,
Joarder2014, Mukherjee2014}.

IAST is one of the available methods for direct macroscopic calculations and
predictions, taking macroscopic data in the form of pure component adsorption
isotherms and predicting macroscopic co-adsorption isotherms. In practice, using
IAST is akin to working in the Grand-Canonical ensemble, but we showed in the
previous section that when the adsorption host is flexible, one should use the
Osmotic ensemble. Instead, an alternative method called the Osmotic Framework
Adsorbed Solution Theory (OFAST)\cite{Coudert2009}, based on the Osmotic
ensemble, should be used when structural transitions occur during adsorption.

% For example \citeauthor{Nijem2012} \cite{Nijem2012} reported
% that \emph{"[their] work unveils unexpected hydrocarbon selectivity in a
% flexible metal-organic framework (MOF), based on differences in their gate
% opening pressure."}

In this section, I will compare the results of IAST and OFAST on two sets of
adsorption data from the published literature on gate-opening materials, and
show that the IAST method gives unrealistic results: it does not reproduce the
gate-opening behavior upon mixture adsorption, and overestimates the selectivity
by up to two orders of magnitude. All of this work is published in
\citejournal{Fraux2018}\cite{Fraux2018}.

\subsection{Ideal Adsorbed Solution Theory}

The Ideal Adsorbed Solution Theory (IAST) starts by assuming that for a given
adsorbent and at fixed temperature $T$, the pure-component isotherms $n_i(P)$
for each gas $i$ of interest is known. Then, given a mixture of ideal gases
adsorbing at total pressure $P$ in an host framework and the composition of the
gas phases $y_i$ --- such that the partial pressures follow the ideal mixing law
$P_i = y_i P$ --- the goal of the method is to predict the total adsorbed
quantity $n_\text{tot}$ and the molar fractions $x_i$ in the adsorbed phase.

In order to do so, \citeauthor{Myers1965}\cite{Myers1965} introduced for each
mixture component a quantity homogeneous to a pressure, $P_i^*$. The IAST method
links this pressure to the compositions of the gas and adsorbed phases by
defining a link between $P_i^*$ and the known variables:
\[P y_i = P_i^* x_i ;\label{eq:spreading}\]
and by imposing the equality of chemical potentials at thermodynamic equilibrium:
\[\label{eq:chem-pot} \forall i,j \quad \int_0^{P_i^*} \frac{n_i(p)}{p} \d p = \int_0^{P_j^*} \frac{n_j(p)}{p} \d p .\]

In the simpler case of two-component gas mixture containing gas B and C, these
two equations and the conservation of matter can be rewritten as a set of four
equations:
\[ P y_B = P_B^* x_B \]
\[ x_B = \frac{P_C^* - P}{P_C^* - P_B^*} \]
\[\frac1{n_\text{tot}} = \frac{x_B}{n_B(P_B^*)} + \frac{1 - x_B}{n_C(P_C^*)}\]
\[ \label{eq:iast}\int_0^{P_B^*} \frac{n_B(p)}{p} \d p = \int_0^{P_C^*} \frac{n_C(p)}{p} \d p\]

Solving these equations for $P_B^*$ and $P_C^*$ will give all the information on
the system composition. It can be done with either numerical integration of the
isotherms, or by fitting the isotherms to a model, and then integrating the
model analytically.

% The IAST model for the prediction of co-adsorption of mixtures in nanoporous
% materials is no panacea, and more involved theories have been developed for
% cases where ideality cannot be assumed: non-ideal adsorbed solution
% models\cite{Yang1987, Sweatman2002} the vacancy solution theory
% (VST),\cite{Suwanayuen1980} etc. However, IAST has been extensively studied and
% both its areas of validity and its weaknesses have been well assessed. In
% particular, it is known to be fairly reliable for adsorption of small gas
% molecules, or mixtures of apolar fluids of a similar chemical nature (such as
% mixtures of hydrocarbons). However, one limitation is that if there are big
% differences in the sorption capacity, extrapolations to high pressures are
% necessary and thus, the resulting mixture behavior predicted can be far off.

\subsection{IAST and flexible frameworks}

The original derivation of the IAST equations\cite{Myers1965} by
\citeauthor{Myers1965} highlights three hypotheses on the co-adsorption process,
on which the model is built:
\begin{description}
    \item[(h1)] The adsorbing framework is inert from a thermodynamic point of view;
    \item[(h2)] The adsorbing framework specific area is constant with respect to
                temperature and the same for all adsorbed species;
    \item[(h3)] The Gibbs definition of adsorption applies.
\end{description}

While the meaning of the last assumption \textbf{(H3)} has been diversely
interpreted by different authors, Myers\cite{Myers1965} originally meant and
later confirmed\cite{Myers2014} it to qualify the method by which the adsorption
isotherms are measured. There is, however, consensus on the fact that absolute
adsorption should be used in IAST calculations --- as opposed to excess or net
adsorption\cite{Brandani2016}. This assumption thus applies equally to both
rigid and flexible adsorbents. However, the first two hypothesis are not valid
for flexible nanoporous materials. \textbf{(H2)} is clearly invalid, as
modifications in both the host's volume and internal structure lead to
variations of pore size and specific area upon structural transitions. We note
here, in passing, that \textbf{(H2)} should already be ruled out for systems of
pore size close to the adsorbate diameter, as well as gas mixtures of widely
different size or shape.  It should, for example, not apply to molecular sieves
systems, yet those can often be described reasonably well by IAST in practice.
Finally, \textbf{(H1)} is violated by all the systems that feature
adsorption-induced deformation, and in particular by systems presenting a
gate-opening or a breathing behavior. As a conclusion, IAST has no theoretical
foundation for those systems and should not be used for co-adsorption prediction
in flexible frameworks.

\begin{figure}[htp]
    \centering
    \input{figures/open-close-selectivity.tex}
    \caption{Typical single-component isotherms for adsorption of two gases (red
    and blue) in a material with gate opening. The gate opening pressure is not
    the same for the two adsorbates, creating a pressure range with a high
    difference in the adsorption capacity for single-components isotherms (gray
    zone in the figure). Contrary to intuition, selectivity will not necessarily
    by high in this pressure range, but will depend on difference in saturation
    uptake $\Delta n$.}
    \label{fig:open-close-selectivity}
\end{figure}

Aside from the mathematical treatment and thermodynamic hypotheses, we can show
in an qualitative way why it is not possible, in flexible host frameworks, to
use the single-component isotherm directly to predict multi-components
adsorption. We address here a common misconception, due to an invalid graphical
interpretation of the isotherms. Figure~\ref{fig:open-close-selectivity} depicts
the equilibrium adsorption isotherms for two different guests in a material
presenting a gate-opening behavior. The gate opening is an adsorption-induced
structural transition from a nonporous to a porous phase of the host, leading to
a step in the single-component adsorption isotherm. Gate opening occurs at two
different pressures for the two adsorbates, due to the specific host--guest
interactions of the two gases (characterized notably by the enthalpy of
adsorption and saturation uptake). In the pressure range in-between the
transition pressures (in gray in figure~\ref{fig:open-close-selectivity}), the
uptake of one species is close to zero --- \emph{in the single-component
isotherm} --- and the uptake of the other species is close to its maximum value.
If these isotherms were encountered for a rigid host material, the selectivity
would be extremely high in this range, with one guest adsorbing but not the
other.

\begin{figure}[htp]
    \centering
    \input{figures/open-close-isotherms.tex}
    \caption{Generation of the total isotherm in gate-opening materials by the
    combination of two single-phase isotherms: an \emph{open} pores isotherm,
    and an \emph{closed} pores isotherm. The transition between the two host
    phases occurs at $P_\text{trans}$.}
    \label{fig:open-close-isotherms}
\end{figure}

Yet, the step in the isotherms here is not simply linked to host--guest
interactions but indeed due to a change in the host structure. In particular,
upon adsorption of a gas mixture in this gate-opening framework, a phase
transition will occur at a given pressure. Before this transition, the structure
will be contracted and show no (or little) adsorption for either guest, and thus
no usable selectivity. After the transition, \emph{both species} will adsorb
into the open pore framework. The selectivity is then governed --- at least
qualitatively --- by the respective saturation uptakes of the two fluids
($\Delta n$ in the figure). While the difference in adsorbed quantities in the
intermediate pressure range visually suggests great selectivity, it is not
possible for one component to adsorb inside the close phase framework while at
the same time the other component adsorb inside the open phase of the framework.
The framework is either in one phase or in the other, at any given time.

The whole issue with using single-component isotherms to predict multi-component
adsorption in frameworks with phase transition boils down to the origin of the
stepped isotherms. The single-component isotherm (represented in
figure~\ref{fig:open-close-isotherms}) is a combination of two isotherms: one in
the first phase (the contracted pore phase), and one in the second phase (the
open pore phase). Both phases --- and the thermodynamic equilibrium between them
--- need to be taken in account to predict the multi-component adsorption.

\subsection{Osmotic Framework Adsorbed Solution Theory}

As established in section~\ref{sec:osmotic-ensemble}, the thermodynamic ensemble
suited for the study of adsorption in flexible materials is the osmotic
ensemble. Let's recall that in this ensemble, the thermodynamic potential
$\Omega$ is a function of the mechanical pressure $P$, the temperature $T$, the
number of atoms in a given host phase $\alpha$ and the adsorbed species chemical
potentials $\mu_i$: \[\Omega(T, P, \mu_i) = F_\alpha + P V_\alpha - \sum_i \mu_i
N_i,\] where $F_\alpha$ is the Helmholtz free energy of the empty host in phase
$\alpha$, $V_\alpha$ the volume of the host in this phase, and $N_i$ the molar
uptake of guest $i$. This expression can be reworked and expressed as a function
not of chemical potentials, but of fluid pressure (taken equal to mechanical
pressure $P$) and adsorption isotherms:\cite{Coudert2008} \[
\label{eq:osmotic-potential}\Omega(T, P, \mu_i) = F_\alpha + P V_\alpha - \sum_i
\int_0^P n_i(T, p) V^m_i(T, p) \ \d p\] Here, $n_i(T,P)$ are the co-adsorption
isotherms for each component and $V^m_i(T,P)$ the molar volume for the species
$i$ in the bulk phase. Supposing that the gases are ideal, the molar volume is
given by $RT/P$, with $R$ the ideal gas constant.

I have shown previously that IAST cannot be used for the study of co-adsorption
in frameworks with adsorption-induced phases transition, because the framework
is not inert during adsorption. However, the IAST assumptions are still valid
for each individual phase of the host matrix, if they are considered in the
absence of a transition. As a consequence, it means that the IAST model can be
used, for each possible host phase $\alpha$, to calculate the co-adsorption
isotherms $n_{\alpha,i}(P,T)$ in this given phase. Then, the thermodynamic
potential of each phase $\Omega_{\alpha}$ can be calculated from these isotherms
through Equation~\eqref{eq:osmotic-potential}, allowing to predict which phase is
the more stable at a given gas phase pressure and composition --- and where the
structural transition(s) occur. This method, extending the IAST theory in the
osmotic ensemble to account for host flexibility, is called Osmotic Framework
Adsorbed Solution Theory (OFAST)\cite{Coudert2009, Coudert2010}.

Although the amount of published data from direct experimental measurements of
co-adsorption of gas mixtures in flexible MOFs is very limited, the OFAST method
has been well validated in the past against experimental data.\cite{Ortiz2011,
Hoffmann2011, Zang2011}. For example, \citeauthor{Ortiz2011}\cite{Ortiz2011}
validated the method against experimental data for adsorption of
\ce{CO2}/\ce{CH4} mixtures in the MIL-53(Al) MOF. MIL-53(Al) is the seminal
example of material with a \emph{breathing} behavior: at low loadings, its most
stable phase is the high porous volume open pores phases, at intermediate
loadings it transition to a closed pores phase, and at high loading it goes back
to the open pores structure. One of the very interesting thing that
\citeauthor{Ortiz2011} predicted using OFAST is the increase of the stability
domain of the closed pore phase when using mixtures with respect to pure
component adsorption. The predicted phase diagrams of MIL-53(Al) is reproduced
in figure~\ref{fig:ofast:ortiz}.

\begin{figure}[ht]
    \centering
    \includegraphics[width=0.8\textwidth]{figures/cited/ofast-phase-diagram-rotated}
    \caption{Temperature–pressure phase diagram of MIL-53(Al) upon adsorption
    of a \ce{CO2}/\ce{CH4} mixture, with increasing \ce{CO2} molar fraction.
    Dashed lines correspond to pure component diagrams.
    Reproduced from reference~\citenum{Ortiz2011} with authorization.\TODO}
    \label{fig:ofast:ortiz}
\end{figure}

In practice, the use of OFAST follows the following steps. First, the host
phases of interest are identified and the single-component adsorption isotherms
$n_{\alpha,i}(T, p)$ for these are obtained: this can be achieved from a fit of
experimental isotherms (see figure~\ref{fig:open-close-isotherms}) or from
molecular simulation.

Secondly, the relative free energies of the host phases (which reduces to a
single $\Delta F_\text{host}$ in our case of two host phases) can be computed
from equation~\eqref{eq:osmotic-potential} and the experimental single-component
stepped isotherm. For example, with two phases $\alpha$ and $\beta$, and
considering ideal gas, we can express equation~\eqref{eq:osmotic-potential} for
each phase:
\[\Omega_\alpha(T, P, \mu_i) = F_\alpha + P V_\alpha - RT \sum_i \int_0^P \frac{n_{\alpha, i}(p)}{p} \ \d p\]
\[\Omega_\beta(T, P, \mu_i) = F_\beta + P V_\beta - RT \sum_i \int_0^P \frac{n_{\beta, i}(p)}{p} \ \d p\]
At the transition ($P=P_\text{trans}$ in figure~\ref{fig:open-close-isotherms},
which is typically known experimentally) the two thermodynamic potentials will
be equal, which gives us a way to evaluate the free energy difference between
the phases:
\[ \label{eq:ofast:delta-f}\Delta F_\text{host} = RT \sum_i \int_0^{P_\text{trans}} \frac{\Delta n_i(T, p)}{p} \d p - P_\text{trans} \Delta V_\text{host}\]

Then, for all values of thermodynamic parameters of interest (pressure and gas
mixture composition) the osmotic potential of the host phases is computed,
enabling the identification of the most stable phase: the phase with the lowest
osmotic potential is the most stable at this pressure and composition. The
pressure at which the osmotic potential in both phases are equal is the phase
transition pressure for a given composition. Finally, we can compute adsorption
properties (guest uptake and selectivity) using IAST in this most stable phase.

\subsection{Comparing IAST and OFAST}

I present here two examples of co-adsorption of gas mixtures in metal-organic
frameworks with gate opening behavior, based on experimental data from the
published literature, comparing the predictions of IAST with those of OFAST. The
first example deals with the adsorption of $\ce{CO2}$, $\ce{CH4}$, and $\ce{O2}$
in the \Cudhbc MOF\cite{Kitaura2003} (see figure~\ref{fig:cu-dhbc}; dhbc =
2,5-dihydroxybenzoate; bpy = bipyridine). These isotherms correspond very
closely to the archetypal \emph{gate opening} scenario described above. The
second example deals with linear alkanes (ethane, propane, and butane)
adsorption in \RPMZn MOF\cite{Nijem2012}; figure~\ref{fig:rpm3-zn} presents the
framework structure of \RPMZn and relevant experimental adsorption and
desorption isotherms.

\subsubsection{Simple isotherms in \Cudhbc}

\begin{figure}[htp]
    \centering
    \raisebox{-0.5\height}{\includegraphics[width=0.35\textwidth]{figures/images/cu-dhbc-structure}}
    \hfill
    \raisebox{-0.5\height}{\input{figures/cu-dhbc.tex}}
    \caption{(left) \Cudhbc structure (from reference.~\cite{Kitaura2003}). (right)
    Sorption isotherms and model isotherms fit at \SI{298}{K} in \Cudhbc for
    various gas compounds. Adsorption data are presented using filled symbols,
    and desorption data using empty symbols. Thick lines are Langmuir isotherms
    fitted at high loading. Experimental data published by
    \citeauthor{Kitaura2003}\cite{Kitaura2003}}
    \label{fig:cu-dhbc}
\end{figure}

\Cudhbc is a textbook example of gate opening upon adsorption, with
single-component adsorption isotherms (reproduced in figure~\ref{fig:cu-dhbc})
that clearly show the transition from a nonporous (at low gas pressure) to a
microporous (at higher pressure) host phase. From the experimental
data\cite{Kitaura2003} I fitted the isotherms at high loading using a Langmuir
model (equation~\eqref{eq:langmuir-isotherm}) for the isotherm in the open pores
structure; and at low loading using a Henry isotherm model
(equation~\eqref{eq:henry-isotherm}) for the closed pores structure.

\[N(p) = K_H \ p \label{eq:henry-isotherm}\]
\[N(p) = N_L \ \frac{K_L \ p}{1 + K_L \ p} \label{eq:langmuir-isotherm}\]

\begin{table}[htp]
    \centering
    \renewcommand{\arraystretch}{1.2}
    \newcolumntype{C}{>{\centering\arraybackslash}X}
    \begin{tabularx}{\textwidth}{c C c C c C}
        \textbf{Gas} & $K_H$ / (mol/atm) & $N_L$ / mol & $K_L$ / atm & $P_\text{trans}$ / atm & $\Delta F$ / (kJ/mol)  \\ \hline
        \ce{CH4}     &     0.0           & 2.86      & 0.134         &                5       & -3.56                  \\
        \ce{CO2}     &     0.0           & 2.79      & 0.699         &                1       & -3.59                  \\
        \ce{O2}      &     0.0           & 2.68      & 0.034         &               20       & -3.39                  \\
    \end{tabularx}
    \caption{Fitted coefficients for the sorption isotherms and free energy
    difference between open and closed structures in \Cudhbc. See
    equations~\eqref{eq:henry-isotherm} and \eqref{eq:langmuir-isotherm} for the
    definitions of $K_H$, $N_L$ and $K_L$.}
    \label{table:cu-dhbc:fit}
\end{table}

The resulting fit parameters are in table~\ref{table:cu-dhbc:fit}. In the
closed-pores phase, I assumed that no adsorption takes place in the whole
pressure range, this is why the $K_H$ coefficient is always zero. Using these
parameters and equation~\eqref{eq:ofast:delta-f}, I computed the free energy
difference for all the isotherms. I took the value of $-3.5 \pm
\SI{0.1}{kJ/mol}$ as the free energy difference between the phases.

\begin{figure}[htp]
    \centering
    % GNUPLOT: LaTeX picture with Postscript
\begingroup
  \makeatletter
  \providecommand\color[2][]{%
    \GenericError{(gnuplot) \space\space\space\@spaces}{%
      Package color not loaded in conjunction with
      terminal option `colourtext'%
    }{See the gnuplot documentation for explanation.%
    }{Either use 'blacktext' in gnuplot or load the package
      color.sty in LaTeX.}%
    \renewcommand\color[2][]{}%
  }%
  \providecommand\includegraphics[2][]{%
    \GenericError{(gnuplot) \space\space\space\@spaces}{%
      Package graphicx or graphics not loaded%
    }{See the gnuplot documentation for explanation.%
    }{The gnuplot epslatex terminal needs graphicx.sty or graphics.sty.}%
    \renewcommand\includegraphics[2][]{}%
  }%
  \providecommand\rotatebox[2]{#2}%
  \@ifundefined{ifGPcolor}{%
    \newif\ifGPcolor
    \GPcolortrue
  }{}%
  \@ifundefined{ifGPblacktext}{%
    \newif\ifGPblacktext
    \GPblacktextfalse
  }{}%
  % define a \g@addto@macro without @ in the name:
  \let\gplgaddtomacro\g@addto@macro
  % define empty templates for all commands taking text:
  \gdef\gplbacktext{}%
  \gdef\gplfronttext{}%
  \makeatother
  \ifGPblacktext
    % no textcolor at all
    \def\colorrgb#1{}%
    \def\colorgray#1{}%
  \else
    % gray or color?
    \ifGPcolor
      \def\colorrgb#1{\color[rgb]{#1}}%
      \def\colorgray#1{\color[gray]{#1}}%
      \expandafter\def\csname LTw\endcsname{\color{white}}%
      \expandafter\def\csname LTb\endcsname{\color{black}}%
      \expandafter\def\csname LTa\endcsname{\color{black}}%
      \expandafter\def\csname LT0\endcsname{\color[rgb]{1,0,0}}%
      \expandafter\def\csname LT1\endcsname{\color[rgb]{0,1,0}}%
      \expandafter\def\csname LT2\endcsname{\color[rgb]{0,0,1}}%
      \expandafter\def\csname LT3\endcsname{\color[rgb]{1,0,1}}%
      \expandafter\def\csname LT4\endcsname{\color[rgb]{0,1,1}}%
      \expandafter\def\csname LT5\endcsname{\color[rgb]{1,1,0}}%
      \expandafter\def\csname LT6\endcsname{\color[rgb]{0,0,0}}%
      \expandafter\def\csname LT7\endcsname{\color[rgb]{1,0.3,0}}%
      \expandafter\def\csname LT8\endcsname{\color[rgb]{0.5,0.5,0.5}}%
    \else
      % gray
      \def\colorrgb#1{\color{black}}%
      \def\colorgray#1{\color[gray]{#1}}%
      \expandafter\def\csname LTw\endcsname{\color{white}}%
      \expandafter\def\csname LTb\endcsname{\color{black}}%
      \expandafter\def\csname LTa\endcsname{\color{black}}%
      \expandafter\def\csname LT0\endcsname{\color{black}}%
      \expandafter\def\csname LT1\endcsname{\color{black}}%
      \expandafter\def\csname LT2\endcsname{\color{black}}%
      \expandafter\def\csname LT3\endcsname{\color{black}}%
      \expandafter\def\csname LT4\endcsname{\color{black}}%
      \expandafter\def\csname LT5\endcsname{\color{black}}%
      \expandafter\def\csname LT6\endcsname{\color{black}}%
      \expandafter\def\csname LT7\endcsname{\color{black}}%
      \expandafter\def\csname LT8\endcsname{\color{black}}%
    \fi
  \fi
    \setlength{\unitlength}{0.0500bp}%
    \ifx\gptboxheight\undefined%
      \newlength{\gptboxheight}%
      \newlength{\gptboxwidth}%
      \newsavebox{\gptboxtext}%
    \fi%
    \setlength{\fboxrule}{0.5pt}%
    \setlength{\fboxsep}{1pt}%
\begin{picture}(7360.00,6800.00)%
    \gplgaddtomacro\gplbacktext{%
      \csname LTb\endcsname%%
      \put(747,4971){\makebox(0,0)[r]{\strut{}$0$}}%
      \csname LTb\endcsname%%
      \put(747,5628){\makebox(0,0)[r]{\strut{}$1000$}}%
      \csname LTb\endcsname%%
      \put(747,6285){\makebox(0,0)[r]{\strut{}$2000$}}%
      \csname LTb\endcsname%%
      \put(849,4719){\makebox(0,0){\strut{}$0$}}%
      \csname LTb\endcsname%%
      \put(1480,4719){\makebox(0,0){\strut{}$20$}}%
      \csname LTb\endcsname%%
      \put(2111,4719){\makebox(0,0){\strut{}$40$}}%
      \csname LTb\endcsname%%
      \put(2742,4719){\makebox(0,0){\strut{}$60$}}%
      \csname LTb\endcsname%%
      \put(3373,4719){\makebox(0,0){\strut{}$80$}}%
    }%
    \gplgaddtomacro\gplfronttext{%
      \csname LTb\endcsname%%
      \put(153,5759){\rotatebox{-270}{\makebox(0,0){\strut{}\ce{CO2} / \ce{O2} selectivity}}}%
      \csname LTb\endcsname%%
      \put(2585,6418){\makebox(0,0)[r]{\strut{}$y_{\ce{CO2}} = 0.1$}}%
      \csname LTb\endcsname%%
      \put(2585,6177){\makebox(0,0)[r]{\strut{}$y_{\ce{CO2}} = 0.5$}}%
      \csname LTb\endcsname%%
      \put(2585,5936){\makebox(0,0)[r]{\strut{}$y_{\ce{CO2}} = 0.9$}}%
    }%
    \gplgaddtomacro\gplbacktext{%
      \csname LTb\endcsname%%
      \put(4241,4905){\makebox(0,0)[r]{\strut{}$1$}}%
      \csname LTb\endcsname%%
      \put(4241,5343){\makebox(0,0)[r]{\strut{}$10$}}%
      \csname LTb\endcsname%%
      \put(4241,5780){\makebox(0,0)[r]{\strut{}$100$}}%
      \csname LTb\endcsname%%
      \put(4241,6218){\makebox(0,0)[r]{\strut{}$1000$}}%
      \csname LTb\endcsname%%
      \put(4343,4719){\makebox(0,0){\strut{}$0$}}%
      \csname LTb\endcsname%%
      \put(5021,4719){\makebox(0,0){\strut{}$20$}}%
      \csname LTb\endcsname%%
      \put(5698,4719){\makebox(0,0){\strut{}$40$}}%
      \csname LTb\endcsname%%
      \put(6376,4719){\makebox(0,0){\strut{}$60$}}%
      \csname LTb\endcsname%%
      \put(7053,4719){\makebox(0,0){\strut{}$80$}}%
    }%
    \gplgaddtomacro\gplfronttext{%
    }%
    \gplgaddtomacro\gplbacktext{%
      \csname LTb\endcsname%%
      \put(543,2719){\makebox(0,0)[r]{\strut{}$0$}}%
      \csname LTb\endcsname%%
      \put(543,3533){\makebox(0,0)[r]{\strut{}$5$}}%
      \csname LTb\endcsname%%
      \put(543,4347){\makebox(0,0)[r]{\strut{}$10$}}%
      \csname LTb\endcsname%%
      \put(645,2452){\makebox(0,0){\strut{}$0$}}%
      \csname LTb\endcsname%%
      \put(1327,2452){\makebox(0,0){\strut{}$20$}}%
      \csname LTb\endcsname%%
      \put(2009,2452){\makebox(0,0){\strut{}$40$}}%
      \csname LTb\endcsname%%
      \put(2691,2452){\makebox(0,0){\strut{}$60$}}%
      \csname LTb\endcsname%%
      \put(3373,2452){\makebox(0,0){\strut{}$80$}}%
    }%
    \gplgaddtomacro\gplfronttext{%
      \csname LTb\endcsname%%
      \put(153,3492){\rotatebox{-270}{\makebox(0,0){\strut{}\ce{CH4} / \ce{O2} selectivity}}}%
    }%
    \gplgaddtomacro\gplbacktext{%
      \csname LTb\endcsname%%
      \put(4037,2638){\makebox(0,0)[r]{\strut{}$1$}}%
      \csname LTb\endcsname%%
      \put(4037,4347){\makebox(0,0)[r]{\strut{}$10$}}%
      \csname LTb\endcsname%%
      \put(4139,2452){\makebox(0,0){\strut{}$0$}}%
      \csname LTb\endcsname%%
      \put(4868,2452){\makebox(0,0){\strut{}$20$}}%
      \csname LTb\endcsname%%
      \put(5596,2452){\makebox(0,0){\strut{}$40$}}%
      \csname LTb\endcsname%%
      \put(6325,2452){\makebox(0,0){\strut{}$60$}}%
      \csname LTb\endcsname%%
      \put(7053,2452){\makebox(0,0){\strut{}$80$}}%
    }%
    \gplgaddtomacro\gplfronttext{%
    }%
    \gplgaddtomacro\gplbacktext{%
      \csname LTb\endcsname%%
      \put(645,595){\makebox(0,0)[r]{\strut{}$0$}}%
      \csname LTb\endcsname%%
      \put(645,1338){\makebox(0,0)[r]{\strut{}$0.5$}}%
      \csname LTb\endcsname%%
      \put(645,2080){\makebox(0,0)[r]{\strut{}$1$}}%
      \csname LTb\endcsname%%
      \put(747,409){\makebox(0,0){\strut{}$0$}}%
      \csname LTb\endcsname%%
      \put(1404,409){\makebox(0,0){\strut{}$20$}}%
      \csname LTb\endcsname%%
      \put(2060,409){\makebox(0,0){\strut{}$40$}}%
      \csname LTb\endcsname%%
      \put(2717,409){\makebox(0,0){\strut{}$60$}}%
      \csname LTb\endcsname%%
      \put(3373,409){\makebox(0,0){\strut{}$80$}}%
    }%
    \gplgaddtomacro\gplfronttext{%
      \csname LTb\endcsname%%
      \put(153,1337){\rotatebox{-270}{\makebox(0,0){\strut{}\ce{CH4} / \ce{CO2} selectivity}}}%
      \csname LTb\endcsname%%
      \put(2060,130){\makebox(0,0){\strut{}pressure / atm}}%
    }%
    \gplgaddtomacro\gplbacktext{%
      \csname LTb\endcsname%%
      \put(4241,595){\makebox(0,0)[r]{\strut{}$0.01$}}%
      \csname LTb\endcsname%%
      \put(4241,1338){\makebox(0,0)[r]{\strut{}$0.1$}}%
      \csname LTb\endcsname%%
      \put(4241,2080){\makebox(0,0)[r]{\strut{}$1$}}%
      \csname LTb\endcsname%%
      \put(4343,409){\makebox(0,0){\strut{}$0$}}%
      \csname LTb\endcsname%%
      \put(5021,409){\makebox(0,0){\strut{}$20$}}%
      \csname LTb\endcsname%%
      \put(5698,409){\makebox(0,0){\strut{}$40$}}%
      \csname LTb\endcsname%%
      \put(6376,409){\makebox(0,0){\strut{}$60$}}%
      \csname LTb\endcsname%%
      \put(7053,409){\makebox(0,0){\strut{}$80$}}%
    }%
    \gplgaddtomacro\gplfronttext{%
      \csname LTb\endcsname%%
      \put(5698,130){\makebox(0,0){\strut{}pressure / atm}}%
    }%
    \gplbacktext
    \put(0,0){\includegraphics{cu-dhbc-selectivities}}%
    \gplfronttext
  \end{picture}%
\endgroup

    \caption{Comparison of IAST (dashed lines) and OFAST (plain lines)
    adsorption selectivity for \ce{CO2}/\ce{O2} (top); \ce{CH4}/\ce{O2} (middle)
    and \ce{CH4}/\ce{CO2} (bottom) mixtures in \Cudhbc. The same curves are
    presented twice, using linear scale for the $y$ axis on the left panels, and
    logarithmic scale on the right panels.}
    \label{fig:cu-dhbc:iast-ofast:selectivity}
\end{figure}

I performed OFAST calculations using Wolfram Mathematica, the corresponding code
is available in the lab's GitHub repository\cite{fx-citable-data}. I used the
PyIAST Python package for the  pure IAST calculations\cite{Simon2016}. For these
IAST calculations, I did not fit the isotherms to a specific model, but rather
the solved the IAST equations by numerical integration and interpolation between
experimental data points. At partial pressures higher than the last point in the
experimental isotherm, that last point was used as saturation uptake. I computed
both partial loading and selectivity between the different gas in a mixture.

Figure~\ref{fig:cu-dhbc:iast-ofast:selectivity} presents the selectivity
obtained with IAST and OFAST; and figure~\ref{fig:cu-dhbc:iast-ofast:loadings}
shows the partial and total loadings for all the gas combinations. The
adsorption selectivity calculated with OFAST follow what one would expect: at
low pressure, the pores are closed and no gas enter the structure, making the
selectivity ill-defined --- the isotherms at low pressure cannot be fitted and
exploited for calculation of separation.  Then, at a pressure depending on the
composition of the gas phase, the gate opening transition occurs. At pressure
higher than gate opening pressure, the framework is in its open pore form, and
the value of selectivity depends on the relative saturation uptake of the two
phases. The selectivities observed are almost independent of the fluid mixture
composition, they are~$\approx 20$ for \ce{CO2}/\ce{O2} and~$\approx 4$ for
\ce{CH4}/\ce{O2} mixtures.

In stark contrast with this picture, the selectivities calculated by IAST are
clearly non-physical. All selectivity curves present a maximum in the pressure
range where gate opening occurs, with selectivities that can be several orders
of magnitude too high, with for example 2 000 instead of 20 for
\ce{CO2}/\ce{O2}. Even at higher pressure --- above the gate opening pressure
range --- the behavior is not identical to the OFAST calculations, because the
incorrect behavior at low pressure affects IAST directly in the integration of
the isotherms (equation~\eqref{eq:iast}).

\begin{figure}[htp]
    \centering
    % GNUPLOT: LaTeX picture with Postscript
\begingroup
  \makeatletter
  \providecommand\color[2][]{%
    \GenericError{(gnuplot) \space\space\space\@spaces}{%
      Package color not loaded in conjunction with
      terminal option `colourtext'%
    }{See the gnuplot documentation for explanation.%
    }{Either use 'blacktext' in gnuplot or load the package
      color.sty in LaTeX.}%
    \renewcommand\color[2][]{}%
  }%
  \providecommand\includegraphics[2][]{%
    \GenericError{(gnuplot) \space\space\space\@spaces}{%
      Package graphicx or graphics not loaded%
    }{See the gnuplot documentation for explanation.%
    }{The gnuplot epslatex terminal needs graphicx.sty or graphics.sty.}%
    \renewcommand\includegraphics[2][]{}%
  }%
  \providecommand\rotatebox[2]{#2}%
  \@ifundefined{ifGPcolor}{%
    \newif\ifGPcolor
    \GPcolortrue
  }{}%
  \@ifundefined{ifGPblacktext}{%
    \newif\ifGPblacktext
    \GPblacktextfalse
  }{}%
  % define a \g@addto@macro without @ in the name:
  \let\gplgaddtomacro\g@addto@macro
  % define empty templates for all commands taking text:
  \gdef\gplbacktext{}%
  \gdef\gplfronttext{}%
  \makeatother
  \ifGPblacktext
    % no textcolor at all
    \def\colorrgb#1{}%
    \def\colorgray#1{}%
  \else
    % gray or color?
    \ifGPcolor
      \def\colorrgb#1{\color[rgb]{#1}}%
      \def\colorgray#1{\color[gray]{#1}}%
      \expandafter\def\csname LTw\endcsname{\color{white}}%
      \expandafter\def\csname LTb\endcsname{\color{black}}%
      \expandafter\def\csname LTa\endcsname{\color{black}}%
      \expandafter\def\csname LT0\endcsname{\color[rgb]{1,0,0}}%
      \expandafter\def\csname LT1\endcsname{\color[rgb]{0,1,0}}%
      \expandafter\def\csname LT2\endcsname{\color[rgb]{0,0,1}}%
      \expandafter\def\csname LT3\endcsname{\color[rgb]{1,0,1}}%
      \expandafter\def\csname LT4\endcsname{\color[rgb]{0,1,1}}%
      \expandafter\def\csname LT5\endcsname{\color[rgb]{1,1,0}}%
      \expandafter\def\csname LT6\endcsname{\color[rgb]{0,0,0}}%
      \expandafter\def\csname LT7\endcsname{\color[rgb]{1,0.3,0}}%
      \expandafter\def\csname LT8\endcsname{\color[rgb]{0.5,0.5,0.5}}%
    \else
      % gray
      \def\colorrgb#1{\color{black}}%
      \def\colorgray#1{\color[gray]{#1}}%
      \expandafter\def\csname LTw\endcsname{\color{white}}%
      \expandafter\def\csname LTb\endcsname{\color{black}}%
      \expandafter\def\csname LTa\endcsname{\color{black}}%
      \expandafter\def\csname LT0\endcsname{\color{black}}%
      \expandafter\def\csname LT1\endcsname{\color{black}}%
      \expandafter\def\csname LT2\endcsname{\color{black}}%
      \expandafter\def\csname LT3\endcsname{\color{black}}%
      \expandafter\def\csname LT4\endcsname{\color{black}}%
      \expandafter\def\csname LT5\endcsname{\color{black}}%
      \expandafter\def\csname LT6\endcsname{\color{black}}%
      \expandafter\def\csname LT7\endcsname{\color{black}}%
      \expandafter\def\csname LT8\endcsname{\color{black}}%
    \fi
  \fi
    \setlength{\unitlength}{0.0500bp}%
    \ifx\gptboxheight\undefined%
      \newlength{\gptboxheight}%
      \newlength{\gptboxwidth}%
      \newsavebox{\gptboxtext}%
    \fi%
    \setlength{\fboxrule}{0.5pt}%
    \setlength{\fboxsep}{1pt}%
\begin{picture}(7580.00,9060.00)%
    \gplgaddtomacro\gplbacktext{%
      \csname LTb\endcsname%%
      \put(860,6354){\makebox(0,0)[r]{\strut{}\small 0}}%
      \csname LTb\endcsname%%
      \put(860,6981){\makebox(0,0)[r]{\strut{}\small 1}}%
      \csname LTb\endcsname%%
      \put(860,7607){\makebox(0,0)[r]{\strut{}\small 2}}%
      \csname LTb\endcsname%%
      \put(860,8234){\makebox(0,0)[r]{\strut{}\small 3}}%
      \csname LTb\endcsname%%
      \put(962,6168){\makebox(0,0){\strut{}\small 0}}%
      \csname LTb\endcsname%%
      \put(1669,6168){\makebox(0,0){\strut{}\small 20}}%
      \csname LTb\endcsname%%
      \put(2376,6168){\makebox(0,0){\strut{}\small 40}}%
      \csname LTb\endcsname%%
      \put(3082,6168){\makebox(0,0){\strut{}\small 60}}%
      \csname LTb\endcsname%%
      \put(3789,6168){\makebox(0,0){\strut{}\small 80}}%
      \csname LTb\endcsname%%
      \put(1971,8878){\makebox(0,0)[l]{\strut{}OFAST}}%
      \csname LTb\endcsname%%
      \put(5457,8878){\makebox(0,0)[l]{\strut{}IAST}}%
      \csname LTb\endcsname%%
      \put(152,6794){\rotatebox{-270}{\makebox(0,0)[l]{\strut{}\ce{CO2} / \ce{O2}}}}%
      \csname LTb\endcsname%%
      \put(152,4348){\rotatebox{-270}{\makebox(0,0)[l]{\strut{}\ce{CH4} / \ce{O2}}}}%
      \csname LTb\endcsname%%
      \put(152,1268){\rotatebox{-270}{\makebox(0,0)[l]{\strut{}\ce{CH4} / \ce{CO2}}}}%
    }%
    \gplgaddtomacro\gplfronttext{%
      \csname LTb\endcsname%%
      \put(572,7294){\rotatebox{-270}{\makebox(0,0){\strut{}uptake / (mol/mol)}}}%
      \colorrgb{0.58,0.00,0.83}%%
      \put(1899,8487){\makebox(0,0)[r]{\strut{}\footnotesize $y_\smallce{CO2} \kern-0.5ex= 0.1$}}%
      \colorrgb{0.00,0.62,0.45}%%
      \put(2715,8487){\makebox(0,0)[r]{\strut{}\footnotesize ~~$y_\smallce{CO2} \kern-0.5ex= 0.5$}}%
      \colorrgb{0.00,0.45,0.70}%%
      \put(3531,8487){\makebox(0,0)[r]{\strut{}\footnotesize ~~$y_\smallce{CO2} \kern-0.5ex= 0.9$}}%
    }%
    \gplgaddtomacro\gplbacktext{%
      \csname LTb\endcsname%%
      \put(4271,6354){\makebox(0,0)[r]{\strut{}\small 0}}%
      \csname LTb\endcsname%%
      \put(4271,6981){\makebox(0,0)[r]{\strut{}\small 1}}%
      \csname LTb\endcsname%%
      \put(4271,7607){\makebox(0,0)[r]{\strut{}\small 2}}%
      \csname LTb\endcsname%%
      \put(4271,8234){\makebox(0,0)[r]{\strut{}\small 3}}%
      \csname LTb\endcsname%%
      \put(4373,6168){\makebox(0,0){\strut{}\small 0}}%
      \csname LTb\endcsname%%
      \put(5080,6168){\makebox(0,0){\strut{}\small 20}}%
      \csname LTb\endcsname%%
      \put(5787,6168){\makebox(0,0){\strut{}\small 40}}%
      \csname LTb\endcsname%%
      \put(6493,6168){\makebox(0,0){\strut{}\small 60}}%
      \csname LTb\endcsname%%
      \put(7200,6168){\makebox(0,0){\strut{}\small 80}}%
      \csname LTb\endcsname%%
      \put(1971,8878){\makebox(0,0)[l]{\strut{}OFAST}}%
      \csname LTb\endcsname%%
      \put(5457,8878){\makebox(0,0)[l]{\strut{}IAST}}%
      \csname LTb\endcsname%%
      \put(152,6794){\rotatebox{-270}{\makebox(0,0)[l]{\strut{}\ce{CO2} / \ce{O2}}}}%
      \csname LTb\endcsname%%
      \put(152,4348){\rotatebox{-270}{\makebox(0,0)[l]{\strut{}\ce{CH4} / \ce{O2}}}}%
      \csname LTb\endcsname%%
      \put(152,1268){\rotatebox{-270}{\makebox(0,0)[l]{\strut{}\ce{CH4} / \ce{CO2}}}}%
    }%
    \gplgaddtomacro\gplfronttext{%
      \colorrgb{0.00,0.00,0.00}%%
      \put(5338,8487){\makebox(0,0)[r]{\strut{}$n_\smallce{CO2}$}}%
      \colorrgb{0.00,0.00,0.00}%%
      \put(6194,8487){\makebox(0,0)[r]{\strut{}~~~~~~~~$n_\smallce{O2}$}}%
    }%
    \gplgaddtomacro\gplbacktext{%
      \csname LTb\endcsname%%
      \put(860,3636){\makebox(0,0)[r]{\strut{}\small 0}}%
      \csname LTb\endcsname%%
      \put(860,4263){\makebox(0,0)[r]{\strut{}\small 1}}%
      \csname LTb\endcsname%%
      \put(860,4889){\makebox(0,0)[r]{\strut{}\small 2}}%
      \csname LTb\endcsname%%
      \put(860,5516){\makebox(0,0)[r]{\strut{}\small 3}}%
      \csname LTb\endcsname%%
      \put(962,3450){\makebox(0,0){\strut{}\small 0}}%
      \csname LTb\endcsname%%
      \put(1669,3450){\makebox(0,0){\strut{}\small 20}}%
      \csname LTb\endcsname%%
      \put(2376,3450){\makebox(0,0){\strut{}\small 40}}%
      \csname LTb\endcsname%%
      \put(3082,3450){\makebox(0,0){\strut{}\small 60}}%
      \csname LTb\endcsname%%
      \put(3789,3450){\makebox(0,0){\strut{}\small 80}}%
      \csname LTb\endcsname%%
      \put(1971,8878){\makebox(0,0)[l]{\strut{}OFAST}}%
      \csname LTb\endcsname%%
      \put(5457,8878){\makebox(0,0)[l]{\strut{}IAST}}%
      \csname LTb\endcsname%%
      \put(152,6794){\rotatebox{-270}{\makebox(0,0)[l]{\strut{}\ce{CO2} / \ce{O2}}}}%
      \csname LTb\endcsname%%
      \put(152,4348){\rotatebox{-270}{\makebox(0,0)[l]{\strut{}\ce{CH4} / \ce{O2}}}}%
      \csname LTb\endcsname%%
      \put(152,1268){\rotatebox{-270}{\makebox(0,0)[l]{\strut{}\ce{CH4} / \ce{CO2}}}}%
    }%
    \gplgaddtomacro\gplfronttext{%
      \csname LTb\endcsname%%
      \put(572,4576){\rotatebox{-270}{\makebox(0,0){\strut{}uptake / (mol/mol)}}}%
      \colorrgb{0.58,0.00,0.83}%%
      \put(1865,5769){\makebox(0,0)[r]{\strut{}\footnotesize $y_\smallce{CH4} \kern-0.5ex= 0.1$}}%
      \colorrgb{0.00,0.62,0.45}%%
      \put(2749,5769){\makebox(0,0)[r]{\strut{}\footnotesize ~~~~$y_\smallce{CH4} \kern-0.5ex= 0.5$}}%
      \colorrgb{0.00,0.45,0.70}%%
      \put(3633,5769){\makebox(0,0)[r]{\strut{}\footnotesize ~~~~$y_\smallce{CH4} \kern-0.5ex= 0.9$}}%
    }%
    \gplgaddtomacro\gplbacktext{%
      \csname LTb\endcsname%%
      \put(4271,3636){\makebox(0,0)[r]{\strut{}\small 0}}%
      \csname LTb\endcsname%%
      \put(4271,4263){\makebox(0,0)[r]{\strut{}\small 1}}%
      \csname LTb\endcsname%%
      \put(4271,4889){\makebox(0,0)[r]{\strut{}\small 2}}%
      \csname LTb\endcsname%%
      \put(4271,5516){\makebox(0,0)[r]{\strut{}\small 3}}%
      \csname LTb\endcsname%%
      \put(4373,3450){\makebox(0,0){\strut{}\small 0}}%
      \csname LTb\endcsname%%
      \put(5080,3450){\makebox(0,0){\strut{}\small 20}}%
      \csname LTb\endcsname%%
      \put(5787,3450){\makebox(0,0){\strut{}\small 40}}%
      \csname LTb\endcsname%%
      \put(6493,3450){\makebox(0,0){\strut{}\small 60}}%
      \csname LTb\endcsname%%
      \put(7200,3450){\makebox(0,0){\strut{}\small 80}}%
      \csname LTb\endcsname%%
      \put(1971,8878){\makebox(0,0)[l]{\strut{}OFAST}}%
      \csname LTb\endcsname%%
      \put(5457,8878){\makebox(0,0)[l]{\strut{}IAST}}%
      \csname LTb\endcsname%%
      \put(152,6794){\rotatebox{-270}{\makebox(0,0)[l]{\strut{}\ce{CO2} / \ce{O2}}}}%
      \csname LTb\endcsname%%
      \put(152,4348){\rotatebox{-270}{\makebox(0,0)[l]{\strut{}\ce{CH4} / \ce{O2}}}}%
      \csname LTb\endcsname%%
      \put(152,1268){\rotatebox{-270}{\makebox(0,0)[l]{\strut{}\ce{CH4} / \ce{CO2}}}}%
    }%
    \gplgaddtomacro\gplfronttext{%
      \colorrgb{0.00,0.00,0.00}%%
      \put(5338,5769){\makebox(0,0)[r]{\strut{}$n_\smallce{CH4}$}}%
      \colorrgb{0.00,0.00,0.00}%%
      \put(6194,5769){\makebox(0,0)[r]{\strut{}~~~~~~~~$n_\smallce{O2}$}}%
    }%
    \gplgaddtomacro\gplbacktext{%
      \csname LTb\endcsname%%
      \put(860,918){\makebox(0,0)[r]{\strut{}\small 0}}%
      \csname LTb\endcsname%%
      \put(860,1545){\makebox(0,0)[r]{\strut{}\small 1}}%
      \csname LTb\endcsname%%
      \put(860,2172){\makebox(0,0)[r]{\strut{}\small 2}}%
      \csname LTb\endcsname%%
      \put(860,2799){\makebox(0,0)[r]{\strut{}\small 3}}%
      \csname LTb\endcsname%%
      \put(962,732){\makebox(0,0){\strut{}\small 0}}%
      \csname LTb\endcsname%%
      \put(1669,732){\makebox(0,0){\strut{}\small 20}}%
      \csname LTb\endcsname%%
      \put(2376,732){\makebox(0,0){\strut{}\small 40}}%
      \csname LTb\endcsname%%
      \put(3082,732){\makebox(0,0){\strut{}\small 60}}%
      \csname LTb\endcsname%%
      \put(3789,732){\makebox(0,0){\strut{}\small 80}}%
      \csname LTb\endcsname%%
      \put(1971,8878){\makebox(0,0)[l]{\strut{}OFAST}}%
      \csname LTb\endcsname%%
      \put(5457,8878){\makebox(0,0)[l]{\strut{}IAST}}%
      \csname LTb\endcsname%%
      \put(152,6794){\rotatebox{-270}{\makebox(0,0)[l]{\strut{}\ce{CO2} / \ce{O2}}}}%
      \csname LTb\endcsname%%
      \put(152,4348){\rotatebox{-270}{\makebox(0,0)[l]{\strut{}\ce{CH4} / \ce{O2}}}}%
      \csname LTb\endcsname%%
      \put(152,1268){\rotatebox{-270}{\makebox(0,0)[l]{\strut{}\ce{CH4} / \ce{CO2}}}}%
    }%
    \gplgaddtomacro\gplfronttext{%
      \csname LTb\endcsname%%
      \put(572,1858){\rotatebox{-270}{\makebox(0,0){\strut{}uptake / (mol/mol)}}}%
      \csname LTb\endcsname%%
      \put(2375,453){\makebox(0,0){\strut{}pressure / atm}}%
      \colorrgb{0.58,0.00,0.83}%%
      \put(1865,3051){\makebox(0,0)[r]{\strut{}\footnotesize $y_\smallce{CH4} \kern-0.5ex= 0.1$}}%
      \colorrgb{0.00,0.62,0.45}%%
      \put(2749,3051){\makebox(0,0)[r]{\strut{}\footnotesize ~~~~$y_\smallce{CH4} \kern-0.5ex= 0.5$}}%
      \colorrgb{0.00,0.45,0.70}%%
      \put(3633,3051){\makebox(0,0)[r]{\strut{}\footnotesize ~~~~$y_\smallce{CH4} \kern-0.5ex= 0.9$}}%
    }%
    \gplgaddtomacro\gplbacktext{%
      \csname LTb\endcsname%%
      \put(4271,918){\makebox(0,0)[r]{\strut{}\small 0}}%
      \csname LTb\endcsname%%
      \put(4271,1545){\makebox(0,0)[r]{\strut{}\small 1}}%
      \csname LTb\endcsname%%
      \put(4271,2172){\makebox(0,0)[r]{\strut{}\small 2}}%
      \csname LTb\endcsname%%
      \put(4271,2799){\makebox(0,0)[r]{\strut{}\small 3}}%
      \csname LTb\endcsname%%
      \put(4373,732){\makebox(0,0){\strut{}\small 0}}%
      \csname LTb\endcsname%%
      \put(5080,732){\makebox(0,0){\strut{}\small 20}}%
      \csname LTb\endcsname%%
      \put(5787,732){\makebox(0,0){\strut{}\small 40}}%
      \csname LTb\endcsname%%
      \put(6493,732){\makebox(0,0){\strut{}\small 60}}%
      \csname LTb\endcsname%%
      \put(7200,732){\makebox(0,0){\strut{}\small 80}}%
      \csname LTb\endcsname%%
      \put(1971,8878){\makebox(0,0)[l]{\strut{}OFAST}}%
      \csname LTb\endcsname%%
      \put(5457,8878){\makebox(0,0)[l]{\strut{}IAST}}%
      \csname LTb\endcsname%%
      \put(152,6794){\rotatebox{-270}{\makebox(0,0)[l]{\strut{}\ce{CO2} / \ce{O2}}}}%
      \csname LTb\endcsname%%
      \put(152,4348){\rotatebox{-270}{\makebox(0,0)[l]{\strut{}\ce{CH4} / \ce{O2}}}}%
      \csname LTb\endcsname%%
      \put(152,1268){\rotatebox{-270}{\makebox(0,0)[l]{\strut{}\ce{CH4} / \ce{CO2}}}}%
    }%
    \gplgaddtomacro\gplfronttext{%
      \csname LTb\endcsname%%
      \put(5786,453){\makebox(0,0){\strut{}pressure / atm}}%
      \colorrgb{0.00,0.00,0.00}%%
      \put(5338,3051){\makebox(0,0)[r]{\strut{}$n_\smallce{CH4}$}}%
      \colorrgb{0.00,0.00,0.00}%%
      \put(6228,3051){\makebox(0,0)[r]{\strut{}~~~~~~~~$n_\smallce{CO2}$}}%
    }%
    \gplbacktext
    \put(0,0){\includegraphics{cu-dhbc-loadings}}%
    \gplfronttext
  \end{picture}%
\endgroup

    \caption{Total (full lines) and partial (dashed lines) loading as function
    of pressure in \Cudhbc for all the gas combination. Left is OFAST results,
    and right correspond to IAST results.}
    \label{fig:cu-dhbc:iast-ofast:loadings}
\end{figure}

Moreover, the IAST selectivity for \ce{CO2}/\ce{O2} presents a big jump around
\SI{40}{atm} when $y_{\ce{CO2}} = 0.1$. Looking at the partial loading in
figure~\ref{fig:cu-dhbc:iast-ofast:loadings} top right panel, we can attribute
this jump to an equilibrium displacement: \ce{O2} is replacing \ce{CO2} in the
structure. This shows again the fact that IAST behaves as if the structure was
closed for \ce{O2} while at the same time being open for \ce{CO2} at pressures
lower than \SI{40}{atm}. I thus confirm by a quantitative study that IAST is not
adapted for adsorption in flexible nanoporous materials.

\FloatBarrier
\subsubsection{More complex isotherms: the case of \RPMZn}

\begin{figure}[htp]
    \centering
    \raisebox{-0.5\height}{\includegraphics[width=0.35\textwidth]{figures/images/rpm3-zn-structure}}
    \hfill
    \raisebox{-0.5\height}{\input{figures/rpm3-zn.tex}}
    \caption{(left) \RPMZn structure (from reference~\cite{Lan2009}).
    (right) Sorption isotherms at \SI{298}{K} for short
    alkanes in \RPMZn. Blue circles are for \ce{C2H6}, red triangles for
    \ce{C3H8}, and green squares for \ce{C4H10}. Filled symbols for adsorption,
    empty symbols for desorption. Thick lines are the open and closed phases fit
    of the isotherms. Experimental data published by
    \citeauthor{Nijem2012}\cite{Nijem2012}}
    \label{fig:rpm3-zn}
\end{figure}

We now turn to a second example of gate opening material, \RPMZn\cite{Lan2009},
which presents more complex adsorption--desorption isotherms for short alkanes
(ethane, propane, butane) --- depicted on the right panel of
figure~\ref{fig:rpm3-zn}. While adsorption of \ce{C2H6}, and \ce{C3H8} in this
material display a typical gate opening behavior, with a well-marked single
transition from a nonporous to a microporous phase, the adsorption of \ce{C4H10}
present two steps at \SI{0.01}{atm} and \SI{0.2}{atm}.  There, the first
transition can be attributed to the structural transition (gate opening), but
the second one is of a different nature. Because there is no hysteresis loop for
the second step, and because it occurs for the larger and more anisotropic guest
molecule, it can be attributed to a fluid reorganization (or fluid packing)
transition inside the pores. Because experimental \emph{in situ}
characterization (such as single X-ray diffraction) would be necessary to
definitely affirm the character of this second step, I chose to work in a
reduced pressure range --- although the OFAST method itself works with host
materials with more than two phases. I thus fitted the \ce{C4H10} isotherm using
a Langmuir isotherm for pressures below \SI{0.2}{atm}.  The OFAST selectivity
after this pressure will thus not be quantitatively accurate, but will be
sufficient for the needed physical insight. I also performed tests by computing
the selectivity under the assumption that the second jump is due to fluid
reorganization by using Langmuir-Freundlich isotherms instead of single site
Langmuir isotherm in the open phase, and the selectivity only differs at
pressures higher than \SI{0.2}{atm}. \fbox{add the corresponding figures?}

\begin{table}[htp]
    \renewcommand{\arraystretch}{1.3}
    \newcolumntype{C}{>{\centering\arraybackslash}X}
    \begin{tabularx}{\textwidth}{c C c C c C}
        \textbf{Gas} & $K_H$ / (mol/bar) & $N_L$ / mol & $K_L$ / bar & $P_\text{trans}$ / bar & $\Delta F$ / (kJ/mol)  \\ \hline
        \ce{C2H6}    & 0.905             & 4.82        & 1.74        &            /           &          /             \\
        \ce{C3H8}    & 2.88              & 9.00        & 42.0        &           0.07         &        -30.1           \\
        \ce{C4H10}   & 27.3              & 5.87        & 699         &           0.01         &        -29.8           \\
    \end{tabularx}
    \caption{Fitted coefficients for the sorption isotherms and free energy
    difference between open and closed structures in \RPMZn. See
    equations~\eqref{eq:henry-isotherm} and \eqref{eq:langmuir-isotherm} for the
    definitions of $K_H$, $N_L$ and $K_L$.}
    \label{table:rpm3-zn:fit}
\end{table}

From the \ce{C3H8} and \ce{C4H10} isotherms, I computed the free energy
difference between the nonporous and microporous phases, which I find to be
$\Delta F = -30.0 \pm \SI{0.1}{kJ/mol}$. The details are in
table~\ref{table:rpm3-zn:fit}. I did not use the \ce{C2H6} isotherms for this
purpose, as it has only limited data at high loading (at pressure above 1 bar),
which increases somewhat the uncertainty of the fit. I was still able to fit the
\ce{C2H6} isotherm with a Langmuir model and use it to compute co-adsorption
data, as the free energy difference of the two host phases do not depend on the
gas.

\begin{figure}[htp]
    \centering
    % GNUPLOT: LaTeX picture with Postscript
\begingroup
  \makeatletter
  \providecommand\color[2][]{%
    \GenericError{(gnuplot) \space\space\space\@spaces}{%
      Package color not loaded in conjunction with
      terminal option `colourtext'%
    }{See the gnuplot documentation for explanation.%
    }{Either use 'blacktext' in gnuplot or load the package
      color.sty in LaTeX.}%
    \renewcommand\color[2][]{}%
  }%
  \providecommand\includegraphics[2][]{%
    \GenericError{(gnuplot) \space\space\space\@spaces}{%
      Package graphicx or graphics not loaded%
    }{See the gnuplot documentation for explanation.%
    }{The gnuplot epslatex terminal needs graphicx.sty or graphics.sty.}%
    \renewcommand\includegraphics[2][]{}%
  }%
  \providecommand\rotatebox[2]{#2}%
  \@ifundefined{ifGPcolor}{%
    \newif\ifGPcolor
    \GPcolortrue
  }{}%
  \@ifundefined{ifGPblacktext}{%
    \newif\ifGPblacktext
    \GPblacktextfalse
  }{}%
  % define a \g@addto@macro without @ in the name:
  \let\gplgaddtomacro\g@addto@macro
  % define empty templates for all commands taking text:
  \gdef\gplbacktext{}%
  \gdef\gplfronttext{}%
  \makeatother
  \ifGPblacktext
    % no textcolor at all
    \def\colorrgb#1{}%
    \def\colorgray#1{}%
  \else
    % gray or color?
    \ifGPcolor
      \def\colorrgb#1{\color[rgb]{#1}}%
      \def\colorgray#1{\color[gray]{#1}}%
      \expandafter\def\csname LTw\endcsname{\color{white}}%
      \expandafter\def\csname LTb\endcsname{\color{black}}%
      \expandafter\def\csname LTa\endcsname{\color{black}}%
      \expandafter\def\csname LT0\endcsname{\color[rgb]{1,0,0}}%
      \expandafter\def\csname LT1\endcsname{\color[rgb]{0,1,0}}%
      \expandafter\def\csname LT2\endcsname{\color[rgb]{0,0,1}}%
      \expandafter\def\csname LT3\endcsname{\color[rgb]{1,0,1}}%
      \expandafter\def\csname LT4\endcsname{\color[rgb]{0,1,1}}%
      \expandafter\def\csname LT5\endcsname{\color[rgb]{1,1,0}}%
      \expandafter\def\csname LT6\endcsname{\color[rgb]{0,0,0}}%
      \expandafter\def\csname LT7\endcsname{\color[rgb]{1,0.3,0}}%
      \expandafter\def\csname LT8\endcsname{\color[rgb]{0.5,0.5,0.5}}%
    \else
      % gray
      \def\colorrgb#1{\color{black}}%
      \def\colorgray#1{\color[gray]{#1}}%
      \expandafter\def\csname LTw\endcsname{\color{white}}%
      \expandafter\def\csname LTb\endcsname{\color{black}}%
      \expandafter\def\csname LTa\endcsname{\color{black}}%
      \expandafter\def\csname LT0\endcsname{\color{black}}%
      \expandafter\def\csname LT1\endcsname{\color{black}}%
      \expandafter\def\csname LT2\endcsname{\color{black}}%
      \expandafter\def\csname LT3\endcsname{\color{black}}%
      \expandafter\def\csname LT4\endcsname{\color{black}}%
      \expandafter\def\csname LT5\endcsname{\color{black}}%
      \expandafter\def\csname LT6\endcsname{\color{black}}%
      \expandafter\def\csname LT7\endcsname{\color{black}}%
      \expandafter\def\csname LT8\endcsname{\color{black}}%
    \fi
  \fi
    \setlength{\unitlength}{0.0500bp}%
    \ifx\gptboxheight\undefined%
      \newlength{\gptboxheight}%
      \newlength{\gptboxwidth}%
      \newsavebox{\gptboxtext}%
    \fi%
    \setlength{\fboxrule}{0.5pt}%
    \setlength{\fboxsep}{1pt}%
\begin{picture}(7360.00,2820.00)%
    \gplgaddtomacro\gplbacktext{%
      \csname LTb\endcsname%%
      \put(498,396){\makebox(0,0)[r]{\strut{}\footnotesize$10^{0}$}}%
      \csname LTb\endcsname%%
      \put(498,1080){\makebox(0,0)[r]{\strut{}\footnotesize$10^{1}$}}%
      \csname LTb\endcsname%%
      \put(498,1763){\makebox(0,0)[r]{\strut{}\footnotesize$10^{2}$}}%
      \csname LTb\endcsname%%
      \put(498,2447){\makebox(0,0)[r]{\strut{}\footnotesize$10^{3}$}}%
      \csname LTb\endcsname%%
      \put(566,248){\makebox(0,0){\strut{}\footnotesize$10^{-3}$}}%
      \csname LTb\endcsname%%
      \put(1127,248){\makebox(0,0){\strut{}\footnotesize$10^{-2}$}}%
      \csname LTb\endcsname%%
      \put(1688,248){\makebox(0,0){\strut{}\footnotesize$10^{-1}$}}%
      \csname LTb\endcsname%%
      \put(2249,248){\makebox(0,0){\strut{}\footnotesize$10^{0}$}}%
    }%
    \gplgaddtomacro\gplfronttext{%
      \csname LTb\endcsname%%
      \put(102,1421){\rotatebox{-270}{\makebox(0,0){\strut{}\small selectivity}}}%
      \csname LTb\endcsname%%
      \put(1407,86){\makebox(0,0){\strut{}\small pressure / bar}}%
      \csname LTb\endcsname%%
      \put(1407,2633){\makebox(0,0){\strut{}\ce{C3H8} / \ce{C2H6}}}%
      \csname LTb\endcsname%%
      \put(1382,2305){\makebox(0,0)[r]{\strut{}\footnotesize$y_{\smallce{C3}} = 0.1$}}%
      \csname LTb\endcsname%%
      \put(1382,2119){\makebox(0,0)[r]{\strut{}\footnotesize$y_{\smallce{C3}} = 0.5$}}%
      \csname LTb\endcsname%%
      \put(1382,1933){\makebox(0,0)[r]{\strut{}\footnotesize$y_{\smallce{C3}} = 0.9$}}%
    }%
    \gplgaddtomacro\gplbacktext{%
      \csname LTb\endcsname%%
      \put(2827,396){\makebox(0,0)[r]{\strut{}\footnotesize$10^{0}$}}%
      \csname LTb\endcsname%%
      \put(2827,1422){\makebox(0,0)[r]{\strut{}\footnotesize$10^{1}$}}%
      \csname LTb\endcsname%%
      \put(2827,2447){\makebox(0,0)[r]{\strut{}\footnotesize$10^{2}$}}%
      \csname LTb\endcsname%%
      \put(2895,248){\makebox(0,0){\strut{}\footnotesize$10^{-3}$}}%
      \csname LTb\endcsname%%
      \put(3497,248){\makebox(0,0){\strut{}\footnotesize$10^{-2}$}}%
      \csname LTb\endcsname%%
      \put(4100,248){\makebox(0,0){\strut{}\footnotesize$10^{-1}$}}%
      \csname LTb\endcsname%%
      \put(4702,248){\makebox(0,0){\strut{}\footnotesize$10^{0}$}}%
    }%
    \gplgaddtomacro\gplfronttext{%
      \csname LTb\endcsname%%
      \put(3798,86){\makebox(0,0){\strut{}\small pressure / bar}}%
      \csname LTb\endcsname%%
      \put(3798,2633){\makebox(0,0){\strut{}\ce{C4H10} / \ce{C3H8}}}%
      \csname LTb\endcsname%%
      \put(4381,2305){\makebox(0,0)[r]{\strut{}\footnotesize$y_{\smallce{C4}} = 0.1$}}%
      \csname LTb\endcsname%%
      \put(4381,2119){\makebox(0,0)[r]{\strut{}\footnotesize$y_{\smallce{C4}} = 0.5$}}%
      \csname LTb\endcsname%%
      \put(4381,1933){\makebox(0,0)[r]{\strut{}\footnotesize$y_{\smallce{C4}} = 0.9$}}%
    }%
    \gplgaddtomacro\gplbacktext{%
      \csname LTb\endcsname%%
      \put(5280,396){\makebox(0,0)[r]{\strut{}\footnotesize$10^{1}$}}%
      \csname LTb\endcsname%%
      \put(5280,1080){\makebox(0,0)[r]{\strut{}\footnotesize$10^{2}$}}%
      \csname LTb\endcsname%%
      \put(5280,1763){\makebox(0,0)[r]{\strut{}\footnotesize$10^{3}$}}%
      \csname LTb\endcsname%%
      \put(5280,2447){\makebox(0,0)[r]{\strut{}\footnotesize$10^{4}$}}%
      \csname LTb\endcsname%%
      \put(5348,248){\makebox(0,0){\strut{}\footnotesize$10^{-3}$}}%
      \csname LTb\endcsname%%
      \put(5950,248){\makebox(0,0){\strut{}\footnotesize$10^{-2}$}}%
      \csname LTb\endcsname%%
      \put(6553,248){\makebox(0,0){\strut{}\footnotesize$10^{-1}$}}%
      \csname LTb\endcsname%%
      \put(7155,248){\makebox(0,0){\strut{}\footnotesize$10^{0}$}}%
    }%
    \gplgaddtomacro\gplfronttext{%
      \csname LTb\endcsname%%
      \put(6251,86){\makebox(0,0){\strut{}\small pressure / bar}}%
      \csname LTb\endcsname%%
      \put(6251,2633){\makebox(0,0){\strut{}\ce{C4H10} / \ce{C2H6}}}%
      \csname LTb\endcsname%%
      \put(6499,2305){\makebox(0,0)[r]{\strut{}\footnotesize$y_{\smallce{C4}} = 0.1$}}%
      \csname LTb\endcsname%%
      \put(6499,2119){\makebox(0,0)[r]{\strut{}\footnotesize$y_{\smallce{C4}} = 0.5$}}%
      \csname LTb\endcsname%%
      \put(6499,1933){\makebox(0,0)[r]{\strut{}\footnotesize$y_{\smallce{C4}} = 0.9$}}%
    }%
    \gplbacktext
    \put(0,0){\includegraphics{rpm3-zn-selectivities}}%
    \gplfronttext
  \end{picture}%
\endgroup

    \caption{IAST (dashed lines) vs OFAST (plain lines) adsorption selectivity
    for \ce{C3H8}/\ce{C2H6} (left) and \ce{C4H10}/\ce{C3H8} (right) mixtures
    in \RPMZn at different compositions.}
    \label{fig:rpm3-zn:iast-ofast:selectivity}
\end{figure}

Figure~\ref{fig:rpm3-zn:iast-ofast:selectivity} displays the selectivity curves
obtained with IAST and OFAST for various gas mixtures and compositions in
\RPMZn; and figure~\ref{fig:rpm3-zn:iast-ofast:loadings} shows the partial and
total loading predicted by both methods. Again, the OFAST selectivity curve
follows the expected behavior: it is constant at low loading, where
single-component isotherms follow the Henry model. In this low-pressure region,
adsorption is negligible and the selectivity cannot be exploited in
adsorption-based processes. However, we can see that because IAST is using
numerical integration, it is much more sensitive to details in the
single-component isotherms than the OFAST method, which is based on fits.

\begin{figure}[htp]
    \centering
    % GNUPLOT: LaTeX picture with Postscript
\begingroup
  \makeatletter
  \providecommand\color[2][]{%
    \GenericError{(gnuplot) \space\space\space\@spaces}{%
      Package color not loaded in conjunction with
      terminal option `colourtext'%
    }{See the gnuplot documentation for explanation.%
    }{Either use 'blacktext' in gnuplot or load the package
      color.sty in LaTeX.}%
    \renewcommand\color[2][]{}%
  }%
  \providecommand\includegraphics[2][]{%
    \GenericError{(gnuplot) \space\space\space\@spaces}{%
      Package graphicx or graphics not loaded%
    }{See the gnuplot documentation for explanation.%
    }{The gnuplot epslatex terminal needs graphicx.sty or graphics.sty.}%
    \renewcommand\includegraphics[2][]{}%
  }%
  \providecommand\rotatebox[2]{#2}%
  \@ifundefined{ifGPcolor}{%
    \newif\ifGPcolor
    \GPcolortrue
  }{}%
  \@ifundefined{ifGPblacktext}{%
    \newif\ifGPblacktext
    \GPblacktextfalse
  }{}%
  % define a \g@addto@macro without @ in the name:
  \let\gplgaddtomacro\g@addto@macro
  % define empty templates for all commands taking text:
  \gdef\gplbacktext{}%
  \gdef\gplfronttext{}%
  \makeatother
  \ifGPblacktext
    % no textcolor at all
    \def\colorrgb#1{}%
    \def\colorgray#1{}%
  \else
    % gray or color?
    \ifGPcolor
      \def\colorrgb#1{\color[rgb]{#1}}%
      \def\colorgray#1{\color[gray]{#1}}%
      \expandafter\def\csname LTw\endcsname{\color{white}}%
      \expandafter\def\csname LTb\endcsname{\color{black}}%
      \expandafter\def\csname LTa\endcsname{\color{black}}%
      \expandafter\def\csname LT0\endcsname{\color[rgb]{1,0,0}}%
      \expandafter\def\csname LT1\endcsname{\color[rgb]{0,1,0}}%
      \expandafter\def\csname LT2\endcsname{\color[rgb]{0,0,1}}%
      \expandafter\def\csname LT3\endcsname{\color[rgb]{1,0,1}}%
      \expandafter\def\csname LT4\endcsname{\color[rgb]{0,1,1}}%
      \expandafter\def\csname LT5\endcsname{\color[rgb]{1,1,0}}%
      \expandafter\def\csname LT6\endcsname{\color[rgb]{0,0,0}}%
      \expandafter\def\csname LT7\endcsname{\color[rgb]{1,0.3,0}}%
      \expandafter\def\csname LT8\endcsname{\color[rgb]{0.5,0.5,0.5}}%
    \else
      % gray
      \def\colorrgb#1{\color{black}}%
      \def\colorgray#1{\color[gray]{#1}}%
      \expandafter\def\csname LTw\endcsname{\color{white}}%
      \expandafter\def\csname LTb\endcsname{\color{black}}%
      \expandafter\def\csname LTa\endcsname{\color{black}}%
      \expandafter\def\csname LT0\endcsname{\color{black}}%
      \expandafter\def\csname LT1\endcsname{\color{black}}%
      \expandafter\def\csname LT2\endcsname{\color{black}}%
      \expandafter\def\csname LT3\endcsname{\color{black}}%
      \expandafter\def\csname LT4\endcsname{\color{black}}%
      \expandafter\def\csname LT5\endcsname{\color{black}}%
      \expandafter\def\csname LT6\endcsname{\color{black}}%
      \expandafter\def\csname LT7\endcsname{\color{black}}%
      \expandafter\def\csname LT8\endcsname{\color{black}}%
    \fi
  \fi
    \setlength{\unitlength}{0.0500bp}%
    \ifx\gptboxheight\undefined%
      \newlength{\gptboxheight}%
      \newlength{\gptboxwidth}%
      \newsavebox{\gptboxtext}%
    \fi%
    \setlength{\fboxrule}{0.5pt}%
    \setlength{\fboxsep}{1pt}%
\begin{picture}(7360.00,10200.00)%
    \gplgaddtomacro\gplbacktext{%
      \csname LTb\endcsname%%
      \put(543,7172){\makebox(0,0)[r]{\strut{}$0$}}%
      \csname LTb\endcsname%%
      \put(543,7616){\makebox(0,0)[r]{\strut{}$2$}}%
      \csname LTb\endcsname%%
      \put(543,8060){\makebox(0,0)[r]{\strut{}$4$}}%
      \csname LTb\endcsname%%
      \put(543,8505){\makebox(0,0)[r]{\strut{}$6$}}%
      \csname LTb\endcsname%%
      \put(543,8949){\makebox(0,0)[r]{\strut{}$8$}}%
      \csname LTb\endcsname%%
      \put(543,9393){\makebox(0,0)[r]{\strut{}$10$}}%
      \csname LTb\endcsname%%
      \put(645,6986){\makebox(0,0){\strut{}\footnotesize $10^{-3}$}}%
      \csname LTb\endcsname%%
      \put(1554,6986){\makebox(0,0){\strut{}\footnotesize $10^{-2}$}}%
      \csname LTb\endcsname%%
      \put(2464,6986){\makebox(0,0){\strut{}\footnotesize $10^{-1}$}}%
      \csname LTb\endcsname%%
      \put(3373,6986){\makebox(0,0){\strut{}\footnotesize $10^{0}$}}%
    }%
    \gplgaddtomacro\gplfronttext{%
      \csname LTb\endcsname%%
      \put(153,8282){\rotatebox{-270}{\makebox(0,0){\strut{}uptake / (mol/mol)}}}%
      \csname LTb\endcsname%%
      \put(2009,10064){\makebox(0,0){\strut{}\ce{C4H10} / \ce{C3H8} (OFAST)}}%
      \colorrgb{0.58,0.00,0.83}%%
      \put(1397,9754){\makebox(0,0)[r]{\strut{}\footnotesize $y_\smallce{C4} = 0.1$}}%
      \colorrgb{0.00,0.62,0.45}%%
      \put(2349,9754){\makebox(0,0)[r]{\strut{}\footnotesize $y_\smallce{C4} = 0.5$}}%
      \colorrgb{0.00,0.45,0.70}%%
      \put(3301,9754){\makebox(0,0)[r]{\strut{}\footnotesize $y_\smallce{C4} = 0.9$}}%
    }%
    \gplgaddtomacro\gplbacktext{%
      \csname LTb\endcsname%%
      \put(4037,7172){\makebox(0,0)[r]{\strut{}$0$}}%
      \csname LTb\endcsname%%
      \put(4037,7616){\makebox(0,0)[r]{\strut{}$2$}}%
      \csname LTb\endcsname%%
      \put(4037,8060){\makebox(0,0)[r]{\strut{}$4$}}%
      \csname LTb\endcsname%%
      \put(4037,8505){\makebox(0,0)[r]{\strut{}$6$}}%
      \csname LTb\endcsname%%
      \put(4037,8949){\makebox(0,0)[r]{\strut{}$8$}}%
      \csname LTb\endcsname%%
      \put(4037,9393){\makebox(0,0)[r]{\strut{}$10$}}%
      \csname LTb\endcsname%%
      \put(4139,6986){\makebox(0,0){\strut{}\footnotesize $10^{-3}$}}%
      \csname LTb\endcsname%%
      \put(5110,6986){\makebox(0,0){\strut{}\footnotesize $10^{-2}$}}%
      \csname LTb\endcsname%%
      \put(6082,6986){\makebox(0,0){\strut{}\footnotesize $10^{-1}$}}%
      \csname LTb\endcsname%%
      \put(7053,6986){\makebox(0,0){\strut{}\footnotesize $10^{0}$}}%
    }%
    \gplgaddtomacro\gplfronttext{%
      \csname LTb\endcsname%%
      \put(5596,10064){\makebox(0,0){\strut{}\ce{C4H10} / \ce{C3H8} (IAST)}}%
      \colorrgb{0.00,0.00,0.00}%%
      \put(5114,9754){\makebox(0,0)[r]{\strut{}$n_\smallce{C4}$}}%
      \colorrgb{0.00,0.00,0.00}%%
      \put(5868,9754){\makebox(0,0)[r]{\strut{}$n_\smallce{C3}$}}%
    }%
    \gplgaddtomacro\gplbacktext{%
      \csname LTb\endcsname%%
      \put(543,3772){\makebox(0,0)[r]{\strut{}$0$}}%
      \csname LTb\endcsname%%
      \put(543,4216){\makebox(0,0)[r]{\strut{}$2$}}%
      \csname LTb\endcsname%%
      \put(543,4660){\makebox(0,0)[r]{\strut{}$4$}}%
      \csname LTb\endcsname%%
      \put(543,5105){\makebox(0,0)[r]{\strut{}$6$}}%
      \csname LTb\endcsname%%
      \put(543,5549){\makebox(0,0)[r]{\strut{}$8$}}%
      \csname LTb\endcsname%%
      \put(543,5993){\makebox(0,0)[r]{\strut{}$10$}}%
      \csname LTb\endcsname%%
      \put(645,3586){\makebox(0,0){\strut{}\footnotesize $10^{-3}$}}%
      \csname LTb\endcsname%%
      \put(1554,3586){\makebox(0,0){\strut{}\footnotesize $10^{-2}$}}%
      \csname LTb\endcsname%%
      \put(2464,3586){\makebox(0,0){\strut{}\footnotesize $10^{-1}$}}%
      \csname LTb\endcsname%%
      \put(3373,3586){\makebox(0,0){\strut{}\footnotesize $10^{0}$}}%
    }%
    \gplgaddtomacro\gplfronttext{%
      \csname LTb\endcsname%%
      \put(153,4882){\rotatebox{-270}{\makebox(0,0){\strut{}uptake / (mol/mol)}}}%
      \csname LTb\endcsname%%
      \put(2009,6664){\makebox(0,0){\strut{}\ce{C3H8} / \ce{C2H6} (OFAST)}}%
      \colorrgb{0.58,0.00,0.83}%%
      \put(1397,6354){\makebox(0,0)[r]{\strut{}\footnotesize $y_\smallce{C3} = 0.1$}}%
      \colorrgb{0.00,0.62,0.45}%%
      \put(2349,6354){\makebox(0,0)[r]{\strut{}\footnotesize $y_\smallce{C3} = 0.5$}}%
      \colorrgb{0.00,0.45,0.70}%%
      \put(3301,6354){\makebox(0,0)[r]{\strut{}\footnotesize $y_\smallce{C3} = 0.9$}}%
    }%
    \gplgaddtomacro\gplbacktext{%
      \csname LTb\endcsname%%
      \put(4037,3772){\makebox(0,0)[r]{\strut{}$0$}}%
      \csname LTb\endcsname%%
      \put(4037,4216){\makebox(0,0)[r]{\strut{}$2$}}%
      \csname LTb\endcsname%%
      \put(4037,4660){\makebox(0,0)[r]{\strut{}$4$}}%
      \csname LTb\endcsname%%
      \put(4037,5105){\makebox(0,0)[r]{\strut{}$6$}}%
      \csname LTb\endcsname%%
      \put(4037,5549){\makebox(0,0)[r]{\strut{}$8$}}%
      \csname LTb\endcsname%%
      \put(4037,5993){\makebox(0,0)[r]{\strut{}$10$}}%
      \csname LTb\endcsname%%
      \put(4139,3586){\makebox(0,0){\strut{}\footnotesize $10^{-3}$}}%
      \csname LTb\endcsname%%
      \put(5110,3586){\makebox(0,0){\strut{}\footnotesize $10^{-2}$}}%
      \csname LTb\endcsname%%
      \put(6082,3586){\makebox(0,0){\strut{}\footnotesize $10^{-1}$}}%
      \csname LTb\endcsname%%
      \put(7053,3586){\makebox(0,0){\strut{}\footnotesize $10^{0}$}}%
    }%
    \gplgaddtomacro\gplfronttext{%
      \csname LTb\endcsname%%
      \put(5596,6664){\makebox(0,0){\strut{}\ce{C3H8} / \ce{C2H6} (IAST)}}%
      \colorrgb{0.00,0.00,0.00}%%
      \put(5114,6354){\makebox(0,0)[r]{\strut{}$n_\smallce{C3}$}}%
      \colorrgb{0.00,0.00,0.00}%%
      \put(5868,6354){\makebox(0,0)[r]{\strut{}$n_\smallce{C2}$}}%
    }%
    \gplgaddtomacro\gplbacktext{%
      \csname LTb\endcsname%%
      \put(543,595){\makebox(0,0)[r]{\strut{}$0$}}%
      \csname LTb\endcsname%%
      \put(543,995){\makebox(0,0)[r]{\strut{}$2$}}%
      \csname LTb\endcsname%%
      \put(543,1395){\makebox(0,0)[r]{\strut{}$4$}}%
      \csname LTb\endcsname%%
      \put(543,1794){\makebox(0,0)[r]{\strut{}$6$}}%
      \csname LTb\endcsname%%
      \put(543,2194){\makebox(0,0)[r]{\strut{}$8$}}%
      \csname LTb\endcsname%%
      \put(543,2594){\makebox(0,0)[r]{\strut{}$10$}}%
      \csname LTb\endcsname%%
      \put(645,409){\makebox(0,0){\strut{}\footnotesize $10^{-3}$}}%
      \csname LTb\endcsname%%
      \put(1554,409){\makebox(0,0){\strut{}\footnotesize $10^{-2}$}}%
      \csname LTb\endcsname%%
      \put(2464,409){\makebox(0,0){\strut{}\footnotesize $10^{-1}$}}%
      \csname LTb\endcsname%%
      \put(3373,409){\makebox(0,0){\strut{}\footnotesize $10^{0}$}}%
    }%
    \gplgaddtomacro\gplfronttext{%
      \csname LTb\endcsname%%
      \put(153,1594){\rotatebox{-270}{\makebox(0,0){\strut{}uptake / (mol/mol)}}}%
      \csname LTb\endcsname%%
      \put(2009,130){\makebox(0,0){\strut{}pressure / atm}}%
      \csname LTb\endcsname%%
      \put(2009,3264){\makebox(0,0){\strut{}\ce{C4H10} / \ce{C2H6} (OFAST)}}%
      \colorrgb{0.58,0.00,0.83}%%
      \put(1397,2954){\makebox(0,0)[r]{\strut{}\footnotesize $y_\smallce{C4} = 0.1$}}%
      \colorrgb{0.00,0.62,0.45}%%
      \put(2349,2954){\makebox(0,0)[r]{\strut{}\footnotesize $y_\smallce{C4} = 0.5$}}%
      \colorrgb{0.00,0.45,0.70}%%
      \put(3301,2954){\makebox(0,0)[r]{\strut{}\footnotesize $y_\smallce{C4} = 0.9$}}%
    }%
    \gplgaddtomacro\gplbacktext{%
      \csname LTb\endcsname%%
      \put(4037,595){\makebox(0,0)[r]{\strut{}$0$}}%
      \csname LTb\endcsname%%
      \put(4037,995){\makebox(0,0)[r]{\strut{}$2$}}%
      \csname LTb\endcsname%%
      \put(4037,1395){\makebox(0,0)[r]{\strut{}$4$}}%
      \csname LTb\endcsname%%
      \put(4037,1794){\makebox(0,0)[r]{\strut{}$6$}}%
      \csname LTb\endcsname%%
      \put(4037,2194){\makebox(0,0)[r]{\strut{}$8$}}%
      \csname LTb\endcsname%%
      \put(4037,2594){\makebox(0,0)[r]{\strut{}$10$}}%
      \csname LTb\endcsname%%
      \put(4139,409){\makebox(0,0){\strut{}\footnotesize $10^{-3}$}}%
      \csname LTb\endcsname%%
      \put(5110,409){\makebox(0,0){\strut{}\footnotesize $10^{-2}$}}%
      \csname LTb\endcsname%%
      \put(6082,409){\makebox(0,0){\strut{}\footnotesize $10^{-1}$}}%
      \csname LTb\endcsname%%
      \put(7053,409){\makebox(0,0){\strut{}\footnotesize $10^{0}$}}%
    }%
    \gplgaddtomacro\gplfronttext{%
      \csname LTb\endcsname%%
      \put(5596,130){\makebox(0,0){\strut{}pressure / atm}}%
      \csname LTb\endcsname%%
      \put(5596,3264){\makebox(0,0){\strut{}\ce{C4H10} / \ce{C2H6} (IAST)}}%
      \colorrgb{0.00,0.00,0.00}%%
      \put(5114,2954){\makebox(0,0)[r]{\strut{}$n_\smallce{C4}$}}%
      \colorrgb{0.00,0.00,0.00}%%
      \put(5868,2954){\makebox(0,0)[r]{\strut{}$n_\smallce{C2}$}}%
    }%
    \gplbacktext
    \put(0,0){\includegraphics{rpm3-zn-loadings}}%
    \gplfronttext
  \end{picture}%
\endgroup

    \caption{Total (full lines) and partial (dashed lines) loading as function
    of pressure in \RPMZn for all the gas combination. Left is OFAST results,
    and right correspond to IAST results.}
    \label{fig:rpm3-zn:iast-ofast:loadings}
\end{figure}

OFAST correctly describes the occurrence of gate opening, at a pressure which
depends on mixture composition but is in the range of the pure component gating
pressures. After gate opening, the selectivity jumps to its value in the open
pore framework. \ce{C3H8}/\ce{C2H6} mixtures have a behavior similar to that
observed in \Cudhbc, with a slowly growing (in logarithmic scale) selectivity at
high loading. On the other hand, OFAST selectivity for \ce{C4H10}/\ce{C3H8}
mixture displays a different behavior. The selectivity is lower after the
transition than before, and further decreases as the pressure and loading
increases. This is due to the fact that the single-component isotherms  in the
open pore structure cross, with \ce{C3H8} adsorbing more than \ce{C4H10} for
pressure bigger than \SI{0.03}{bar}. Thus, the low-pressure selectivity is
reversed at high pressure.

In contrast, the IAST fails to describe gate opening, with selectivity showing a
continuous evolution. Even the trends displayed by this evolution are in poor
agreement and make no physical sense, featuring non-monotonic evolution as a
function of pressure and composition. Even their high-pressure limit is often
far off from reality, as seen in the case of \ce{C3H8}/\ce{C2H6}.

\FloatBarrier
\nonumsection{Conclusions}

% Several published studies of fluid mixture co-adsorption in flexible nanoporous
% material use the Ideal Adsorbed Solution Theory (IAST) method to predict the
% co-adsorption behavior based on single-component adsorption isotherms. This is an
% invalid application of IAST, which is not adapted to flexible frameworks, as its
% very first hypothesis is that the framework is inert during adsorption --- as
% clearly stated in the derivation of the method in the seminal IAST
% paper\cite{Myers1965}. However, the IAST method can be adapted for frameworks
% presenting phase transitions induced by adsorption by using the osmotic
% thermodynamic ensemble. This extension of IAST to flexible materials is called
% Osmotic Framework Adsorbed Solution Theory (OFAST)\cite{Coudert2010}. It allows
% the prediction of phases transitions upon co-adsorption, as well as the details
% of the multi-component co-adsorption isotherms, and is available in commercial
% software\cite{VanAssche2016}. Moreover, the use of OFAST with data at various
% temperatures allows one to produce multi-dimensional {temperature, pressure,
% mixture composition} phase diagrams for the flexible host\cite{Ortiz2011}.
% Finally, while OFAST itself relies on the IAST to describe adsorption in each
% phase of the host material, this method of accounting for flexibility is not
% limited to IAST and can be used with other adsorbed solution models, such as
% real adsorbed solution theory (RAST) or vacancy solution theory (VST).

In this chapter, I compared the IAST and OFAST macroscopic methods for the
prediction of co-adsorption of fluid mixtures in two different frameworks
presenting a gate-opening behavior. In both cases, the selectivities derived by
the IAST method are nonphysical and differ widely from the OFAST results, over-
or under-estimate the selectivity, sometimes by up to two orders of magnitude.
Moreover, this show that even without explicitly using IAST for calculations of
selectivity in flexible frameworks, one has to be cautious in comparing
single-component isotherms of different guests. Differences in step pressure of
stepped isotherms can lead to claims of strong selectivity using flexibility,
when applying --- without noticing it --- concepts that are valid only for rigid
host matrices.

Macroscopic methods such as the ones used here are only applicable under
restrictive hypotheses; for example both OFAST and IAST assume an ideal mixture
of perfect gases when deriving and solving the equations. Going past these
hypotheses is possible but requires empirical models for the chemical
potentials. Despite these limitations, they are very useful in the study of
adsorption in flexible materials as they are orders of magnitude faster than
fully atomistic methods and allow for a faster screening and estimations of the
performance of different materials. If we want to overcome the limitations of
such macroscopic methods, we can turn to atomistic simulations. Such simulations
allow to explore non-ideal systems containing real gases without restricting
ourselves to a specific expression for the chemical potentials. In the next
chapter, I am going to present the framework of statistical thermodynamic that
is used to link together atomistic and the macroscopic description of a system.

\OnlyInSubfile{\printbibliography}

\end{document}
