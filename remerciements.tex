%!TEX root = A-web.tex
%!TeX spellcheck = fr-FR
% Guillaume Fraux PhD thesis -- (C) 2019
% Distributed under CC-BY-SA-NC license 4.0
% Creative Commons Attribution-NonCommercial-ShareAlike 4.0 International

\begin{otherlanguage}{french}

% Chapter-style header without adding to the TOC
\hrule\relax
\vspace*{.9\baselineskip}%
\raggedright{\huge\spacedallcaps{Remerciements}}\par%
\vspace*{1.1\baselineskip}%
\hrule\relax
\vspace*{\baselineskip}%
\thispagestyle{empty}

\begingroup
\itshape

En premier lieu, je voudrais adresser ici mes plus vifs remerciements à Sofia
Calero et Paul Fleurat-Lessart pour avoir accepté d'être les rapporteurs de ce
manuscrit, et pour l'intérêt qu'ils ont porté à ce travail. Je remercie aussi
Caroline Mellot-Draznieks, Renaud Denoyel, et Alain Fuchs qui m'ont fait
l'honneur de faire partie de mon jury de thèse.

J'ai eu la chance d'effectuer ma thèse sous la direction de François-Xavier
Coudert, qui m'a beaucoup apporté tant sur le plan scientifique qu'humain. Je
souhaite lui exprimer ma gratitude pour tout ce que j'ai pu apprendre à son
contact, et pour son immense disponibilité. Il m'a non seulement aidé à acquérir
des compétences scientifique variée, allant de la physique statistique à la
chimie en passant par la simulation moléculaire et la programmation; mais aussi
une compréhension du monde de la recherche et du travail de chercheur. Je veux
aussi remercier Anne Boutin et Alain Fuchs pour le temps qu'ils ont consacré à
mon travail, leurs conseils et nos discussions. Je leur dois un intérêt fort
pour la thermodynamique et les méthodes Monte Carlo.

Je remercie aussi tous ceux avec qui j'ai eu l'occasion de travailler à Chimie
ParisTech, qui ont fait du labo un lieu de travail agréable et chaleureux. Merci
à Adrian, Clarisse, Emmanuel, Fabien, Félix, Miguel, Siwar, Srinidhi, Thibaud et
Wenke pour toutes nos discussions, toujours enrichissantes. Merci en particulier
à Jack et Jean-Mathieu pour leur bonne humeur; Romain pour son ouverture sur
l'industrie qui m'a beaucoup apporté; Daniela pour ses conseils et ses qualités
humaines, au labo comme en conférences; Johannes et Laura pour nos
collaborations et discussions scientifiques.

La thèse ça n'avance pas toujours comme on le voudrait, et il est important
d'avoir un lieux où se vider la thèse pour mieux y revenir. Le groupe SGDF de
Montreuil m'a permis de passer du temps au vert avec des jeunes formidables. Je
voudrais remerciés particulièrement Patricia, Florence, les deux Laurent et
Anne-Cécile pour leur soutien et leurs encouragements. Merci aussi à Nathanaël,
Julia, Ophélie, Antonin et Elsa: vous avez été de super co-chefs ! De même, de
grands mercis au club de musique Folk de l'ENS, et en particulier à Martin,
Chloé, Jeremy, Héloïse et Olivier.

Je souhaite aussi remercier ma famille: mes frères et sœurs Béatrice, Paul,
David et Matthieu; ma grand-mère qui m'a apporté une éthique de vie et des
racines; et surtout mes parents. Sans leur soutien et leurs encouragements
constants, je ne serais pas là aujourd'hui.

Enfin, je dois énormément à Margot, qui m'a soutenue pendant cette thèse, et a
su supporter mon humour douteux et mes couchers tardifs. Merci d'apporter dans
ma vie joie et bonne humeur, discussions insensée et profondes, expériences
culinaires et voyages au bout du monde!

\endgroup

\clearpage
\mbox{}
\thispagestyle{empty}
\clearpage

\ifweb

\mbox{}\vfill
\thispagestyle{empty}

Copyright © 2019 Guillaume Fraux

This document is distributed under a Creative Common license CC-BY-SA-NC 4.0
(Creative Commons Attribution-NonCommercial-ShareAlike 4.0 International).

See \url{https://creativecommons.org/licenses/by-nc-sa/4.0/legalcode} for the
full text of the license.

\begin{center}
    \includegraphics[width=15em]{figures/images/by-nc-sa-eu.png}
\end{center}

\clearpage
\mbox{}
\thispagestyle{empty}
\clearpage

\fi

\end{otherlanguage}
