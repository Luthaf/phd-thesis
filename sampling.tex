\documentclass[thesis]{subfiles}

\begin{document}

\OnlyInSubfile{\setcounter{chapter}{2}}

\chapter{Molecular simulations}
\startcontents[chapters]
\printpartialtoc

\section{Statistical thermodynamics}

Macroscopic methods such as the one I discussed in chapter~\ref{sec:macroscopic}
can provide understanding and allow us to predict behavior in some cases; they
are not always sufficiently precise to gain a complete understanding of the
phenomenon at play. In particular, as these methods describe the systems at the
macroscopic level, they don't take into account the individual atoms, and the
interactions between them. Put another way, they don't describe the
\emph{chemistry} of the system. Statistical thermodynamic is a tool that we can
use to bring together the microscopic description of matter and the macroscopic
behavior and characteristics of the system (pressure, temperature, \dots).

In this chapter, I will derive and recall some concepts from statistical
thermodynamics I used during my PhD. For a more in depth description of
statistical mechanics, I recommend the books of
\citeauthor{Tuckerman2010}\cite{Tuckerman2010}, or (in french)
\citeauthor{Diu1996}\cite{Diu1996}.

\subsection{Maxwell-Boltzmann statistics in canonical ensemble}

We will consider a molecular system containing $N$ individual atoms behaving as
classical particles with individual positions $\r_i$, masses $m_i$ and momentum
$\p_i$. Supposing that these particles are in some container of fixed volume
$V$, and at thermal equilibrium with a thermostat at temperature $T$, they
evolve in the canonical or $NVT$ ensemble. Finally, we will also assume that the
atoms in the system follow the Maxwell-Boltzmann statistics, \ie that the
probability that the system is in a state of internal energy $E_i$ is given by:
\[\mathcal{P}_i = \frac 1 Z e^{-\beta E_i}, \label{eq:maxwell-boltzmann}\]
where $\beta = 1 / k_B T$ with $k_B$ the Boltzmann constant and $Z$ is a
normalization constant called the \emph{partition function}. To be more precise,
particles would either follow Bose–Einstein statistics for Bosons (particles
with an integer spin, such as photons) or the Fermi–Dirac statistics for
Fermions (particles with half-integer spin, such as electrons or protons). But
as both Bose-Einstein and Fermi-Dirac statistics reduce to the Maxwell-Boltzmann
distribution when temperature is high enough, we will use this distribution
instead.

We define the state of a system by the values taken by all the positions $\r_i$
and all the momentum $\p_i$ of all the $N$ atoms in the system. The state of the
system is then defined by $6N$ variables, or a point in a vector space of $6N$
dimensions called the \emph{phase space}. In order to compute the energy of a
state, we will describe the interactions between the atoms by a potential energy
$\mathcal{V}(\r^N)$, independent of time. Then, we can compute the total energy
of a state using the classical Hamiltonian of the system:
\[H(\r^N, \p^N) = \sum_i^N \frac{\p_i^2}{2 m_i} + \mathcal{V}(\r^N);\]
where I use $\r^N$ and $\p^N$ as shorthand for the set of all positions
$\{\r_i\}$ and momentum $\{\p_i\}$ respectively.

The last element in equation~\eqref{eq:maxwell-boltzmann} we need to compute is
the partition function $Z$. In order to do so, we can note that the probability
for the system to be anywhere in the phase space $\Phi$ should be 1.
\[\iiint_\Phi \mathcal{P}_i = 1\]
\[Z = \frac{1}{h^{3N}} \iiint_\Phi e^{-\beta H(\r^N, \p^N)} \ \d\r^N \d\p^N\]
where the Planck constant $h$ is used as a normalization factor used to make
sure that $Z$ has the right dimension.

We can already compute at least a part of this integral by separating the
kinetic and potential energy terms in the Hamiltonian:
\[Z = \frac{1}{h^{3N}} \prod_i^{3N} \int e^{-\beta \frac{p_i^2}{2 m_i}} \ \d p_i \iiint_V e^{-\beta \mathcal{V}(\r^N)} \ \d\r^N \]
where the potential energy integral is over all the accessible volume. The
kinetic energy term is a product of Gaussian integrals, and gives us the
following expression for the partition function:
\[Z = \prod_i^{3N} \sqrt{\frac{2\pi m_i}{\beta h^2}} \iiint_V e^{-\beta \mathcal{V}(\r^N)} \ \d\r^N\]
$\lambda_i = \sqrt{\frac{\beta h^2}{2\pi m_i}}$ is the de Broglie thermal
wavelength for a particle with mass $m_i$, and is homogeneous to a distance. In
the following, I will be using $\Lambda^N$ for the product over all particles
$\prod_i^N \lambda_i^3$. This gives the final expression for the partition
function:
\[Z = \frac{1}{\Lambda^N} \iiint_V e^{-\beta \mathcal{V}(\r^N)} \ \d\r^N \label{eq:partition-function}\]

\subsubsection{Thermodynamic quantities from the partition function}

It is possible to use the knowledge of the partition function to compute some
of the macroscopic properties of our system. For examples, the internal energy
is the average value of the Hamiltonian:
\[U = \frac{1}{Z} \iiint_\Phi H e^{-\beta H} \]
If we express $H e^{-\beta H}$ as the partial derivative of $e^{-\beta H}$ with
respect to $\beta$ we get
\[U = \frac{1}{Z} \iiint_\Phi -\frac{\partial}{\partial \beta} e^{-\beta H} \]
\[U = \frac{-1}{Z} \frac{\partial}{\partial \beta} \iiint_\Phi e^{-\beta H} \]
\[U = \frac{-1}{Z} \frac{\partial Z}{\partial \beta}\]
Or equivalently
\[U = \frac{\partial \ln Z}{\partial \beta}\]

Using the same ideas, we can derive expressions\cite{Tuckerman2010} for the
entropy $S$ and the free energy $F$ using the partition function:
\[F = - \frac 1 \beta \ln Z\]
\[S = k_B \left[\ln Z + \beta \frac{\partial \ln Z}{\partial \beta} \right]\]

\subsubsection{Observables}

The probability for the system to be in a given state gives us the missing link
between microscopic and macroscopic properties of the system. We can express a
macroscopic observable property $A$ that we can also compute or measure at a
microscopic level using the the Maxwell-Boltzmann statistics:
\[A = \braket{A} = \iiint_\Phi P_i A_i.\]
The value of $A$ at a macroscopic level is the same value as the ensemble
average $\braket{A}$, which depends on both the value of the property in a given
macroscopic state $A_i$, and the probability of the system to be in this state.
Using equations~\eqref{eq:maxwell-boltzmann} and~\eqref{eq:partition-function}
together, we can express the average value for any observable property in the
canonical ensemble:
\[\braket{A} = \frac{\iiint_\Phi \d\r^N \d\p^N \; A(\r^N, \p^N) \ e^{-\beta H(\r^N, \p^N)}}{\iiint_\Phi \d\r^N \d\p^N \; e^{-\beta H(\r^N, \p^N)}}. \label{eq:observable-micro-macro}\]

\subsubsection{Sampling}

This theoretical approach to define macroscopic properties from microscopic data
is useless unless we can compute the integrals over the whole phase space $\Phi$
in~\eqref{eq:observable-micro-macro}. But computing this integral explicitly in
all but the simplest cases will prove difficult, as the phase space is a $6N$
dimensional vector space; and values for $N$ range from a few hundred all the
way up to Avogadro number. But in general a lot of states in the phase space are
not relevant when computing the integral, mainly because they have a too high
energy. So instead of computing the whole integral, we resort to only use a
finite number of samples in the phase space, which we try to pick as the most
relevant. In a semi-formal manner, we try to generate a set of points $\phi$
inside the phase space, such as
\[\braket{A} \approx \frac{\sum_\phi A(\r^N, \p^N) e^{-\beta H(\r^N, \p^N)}}{\sum_\phi e^{-\beta H(\r^N, \p^N)}}.\]

There are a few algorithms we can use to do this sampling and generate the set
$\phi$ of points we will use to compute a given property. I will discuss two of
them below: the Metropolis Monte Carlo (MC) method, and Molecular Dynamics (MD).
If we can get these algorithms to generate a set of states in the phase space
according to the Maxwell-Boltzmann probability, with the same state appearing
possibly more than once in the set, we can simplify the calculation of ensemble
average of observables even further. For a set of $m$ states indexed by
$\alpha$, the average reads as:
\[\braket{A} \approx \frac 1 m \sum_\alpha^m A(\r_\alpha^N, \p_\alpha^N).\]

\subsection{Other thermodynamic ensembles}

It is possible to show that in other thermodynamic ensembles there is a similar
probability distributions for the phase space, and the associated partition
function is also defined by the normalization of these probability distributions.

\subsubsection{Isothermal-isobaric ensemble}

In the $NPT$ ensemble, the probability for the system to be in a state is given
by:
\[ \mathcal{P}_{NPT} = \frac{1}{\Delta} e^{-\beta \left[H(\r^N, \p^N) \ + \ PV\right]} \]
\[ \Delta = \frac{1}{\Lambda^N} \int \d V \iiint_V \d\r^N \; e^{-\beta \left[\mathcal{V}(\r^N) \ + \ PV\right]} \]

And the free energy is given by:
\[G = - \frac 1 \beta \ln \Delta.\]

\subsubsection{Grand canonical ensemble}

In the $\mu VT$ ensemble, the probability for the system to be in a state is
given by:
\[ \mathcal{P}_{\mu VT} = \frac{1}{\Theta} e^{-\beta \left[H(\r^N, \p^N) \ - \ \sum_i \mu_i n_i \right]} \]
\[ \Theta = \frac{1}{\Lambda^N} \iint \d n_i \iiint_V \d\r^N \d\p^N \; e^{-\beta \left[\mathcal{V}(\r^N) \ - \ \sum_i \mu_i n_i \right]} \]

\subsubsection{Osmotic ensemble}

In the osmotic ($N_\text{host} \ \mu \ PT$) ensemble, the probability for the system to be in
a state is given by:
\[ \mathcal{P}_{\mu VT} = \frac{1}{\zeta} e^{-\beta \left[H(\r^N, \p^N) \ + \ PV \ - \ \sum_i \mu_i n_i \right]} \]
\[ \zeta = \frac{1}{\Lambda^N} \int \d V \iint \d n_i \iiint_V \d\r^N \d\p^N \; e^{-\beta \left[\mathcal{V}(\r^N) \ + \ PV \ - \ \sum_i \mu_i n_i \right]} \]

\section{Metropolis Monte Carlo}
~

\subsection{Direct sampling of ensembles}
~

\subsection{Monte Carlo moves}
~

\subsection{Computing energy: some tricks of the trade}
~

\section{Molecular dynamics}
~
% We will assume that the atoms behave as classical point particles

\subsection{Integrators}
~

\subsection{Sampling other ensembles}
~

\section{Free energy methods}
~

\OnlyInSubfile{\printbibliography}

\end{document}
