\documentclass[11pt, a4paper]{book}
\usepackage[T1]{fontenc}
\usepackage[utf8]{inputenc}
\usepackage{classicthesis}
\usepackage[greek, english]{babel}
\usepackage{libertine}
\usepackage{mathtools}
\usepackage{amssymb}
\usepackage[frenchmath]{mathastext}
\usepackage{dsfont}
\usepackage{subfiles}
\usepackage{graphicx}
\usepackage{tabularx}
\usepackage{microtype}
\usepackage{xspace}
\usepackage{mdframed}
\usepackage{placeins}
\usepackage{booktabs}
\usepackage{afterpage}
\usepackage{xpatch}
\usepackage[parfill]{parskip}
\usepackage[autostyle=false, style=english]{csquotes}
\usepackage[font={it}]{caption}
\usepackage[
    backend=biber,
    bibencoding=utf8,
    maxbibnames=99,
    maxcitenames=1,
    url=false,
    sorting=none,
    citestyle=numeric-comp,
]{biblatex}

\usepackage[version=4]{mhchem}
\usepackage{braket}
\usepackage{siunitx}

\usepackage[outputdir=build]{minted}
\usepackage{inconsolata}

% Automagically transform "foobar" to ``foobar''
\MakeOuterQuote{"}

% No Overfull hbox warning for less than 1pt
\hfuzz=1pt

% Float placement rules, from https://aty.sdsu.edu/bibliog/latex/floats.html
\renewcommand{\topfraction}{0.9}	% max fraction of floats at top
\renewcommand{\bottomfraction}{0.8}	% max fraction of floats at bottom
%  Parameters for TEXT pages (not float pages):
\setcounter{topnumber}{2}
\setcounter{bottomnumber}{2}
\setcounter{totalnumber}{4}
\renewcommand{\dbltopfraction}{0.9}	% fit big float above 2-col. text
\renewcommand{\textfraction}{0.07}	% allow minimal text w. figs
% Parameters for FLOAT pages (not text pages):
\renewcommand{\floatpagefraction}{0.8}	% require fuller float pages
% N.B.: floatpagefraction MUST be less than topfraction !!
\renewcommand{\dblfloatpagefraction}{0.8}	% require fuller float pages

% space left between floats (default is 12.0pt plus 2.0pt minus 2.0pt).
\setlength{\floatsep}{\parskip}
% space left on top and bottom of an in-text float (default is 12.0pt plus 2.0pt minus 2.0pt).
\setlength{\intextsep}{\parskip}
% space between last top float or first bottom float and the text (default is 20.0pt plus 2.0pt minus 4.0pt).
\setlength{\textfloatsep}{2\parskip}

% Recompile the figures as needed
\immediate\write18{cd figures && make}

\graphicspath{{./}{./figures/}}

% Quotations with a markdown style
\surroundwithmdframed[linewidth=2pt,topline=false,bottomline=false,rightline=false]{quote}
\AtBeginEnvironment{quote}{\itshape}

\makeatletter
% Ensure hyperref produce the right links by adding a phantomsection before addcontentsline
\let\@oldaddcontentsline\addcontentsline
\renewcommand{\addcontentsline}{\phantomsection\@oldaddcontentsline}

\let\@oldchapter\chapter
% Skip at least one page before chapter (OnlyInMainfile), and print toc after
\newcommand{\@chapterstar}[1]{
    \OnlyInMainfile{\clearpage\pagestyle{empty}\null}%
    \@oldchapter*{#1}%
    \pagestyle{fancy}%
}
\newcommand{\@chapternostar}[1]{
    \OnlyInMainfile{\clearpage\pagestyle{empty}\null}%
    \@oldchapter{#1}%
    \startcontents[chapters]\printpartialtoc% only print TOC for numbered chapters
    \pagestyle{fancy}
}
\renewcommand{\chapter}{\@ifstar{\@chapterstar}{\@chapternostar}}

% Use the chapter* style for table of contents, but without adding itself
% to the TOC ...
\renewcommand\tableofcontents{%
    \hrule\relax
    \vspace*{.9\baselineskip}%
    \raggedright{\huge\spacedallcaps{\contentsname}}\par%
    \markboth{\contentsname}{\contentsname}
    \vspace*{1.1\baselineskip}%
    \hrule\relax
    \vspace*{\baselineskip}%
    \@starttoc{toc}%
}

% Only expand the argument in a subfile/not in a subfile
\ifdef{\old@document@subfiles}%
    {\def\OnlyInSubfile#1{#1}\def\OnlyInMainfile#1{}}%
    {\def\OnlyInSubfile#1{}\def\OnlyInMainfile#1{#1}}

\let\@oldprintbibliography\printbibliography
\renewcommand{\printbibliography}{
    \begingroup
        % Use initials for author names in the general bibliography
        \makeatletter\toggletrue{abx@bool@giveninits}\makeatother
        % Give more leeway to LaTeX when trying to typeset overfull paragraphs
        \setlength{\emergencystretch}{2em}
        \@oldprintbibliography
    \endgroup
}
\makeatother

% A bibheading using subsections
\defbibheading{subsectionbibheading}{\subsection*{#1}}

\DeclareCiteCommand{\citejournal}
    {\usebibmacro{prenote}}
    {%
        \usebibmacro{journal}%
        \setunit{\addspace(}%
        {\printfield{year})\xspace}%
    }
    {\multicitedelim}
    {\usebibmacro{postnote}}

\DeclareFieldFormat{doi}{%
    \color{CTurl}\small%
    \mkbibacro{DOI}\addcolon\space%
    \ifhyperref%
        {\href{https://doi.org/#1}{\nolinkurl{#1}}}%
        {\nolinkurl{#1}}%
}

% An enumerate like environement for the publication list
\newcounter{pubcount}
\setcounter{pubcount}{0}
\defbibenvironment{publist}
    {
        \begin{enumerate}
        \setcounter{enumi}{\thepubcount}
    }
    {
        \setcounter{pubcount}{\theenumi}
        \end{enumerate}
    }
    {\item}

% Use \[ and \] for all equations, and make all equations numbered
\def\[{\begin{equation}}
\def\]{\end{equation}}

% Partial TOC formatting for chapters
\newcommand{\printpartialtoc}{%
    \begingroup
        \hypersetup{hidelinks}
        \hrule\vspace*{0.6em}%
        \printcontents[chapters]{}{1}{\setlength\parskip{0pt}}%
        \vspace*{0.6em}\hrule
    \endgroup
}

% Add hyphenation rule for bibliography
\hyphenation{Chem-Phys-Chem}
\hyphenation{Geor-geault}

% Convenience macros
%% Math stuff
\let\phi\varphi
\let\epsilon\varepsilon
\let\vec\mathnormalbold
\def\d{\text{d}}
\let\oldring\r
\def\r{\vec{r}}
\def\p{\vec{p}}
\def\v{\vec{v}}
\def\smallo{{\scriptstyle\mathcal{O}}}
\def\erfc{{\normalfont\text{erfc}}}
\def\erf{{\normalfont\text{erf}}}
% Fix formatting of \neq with mathastext
\def\neq{\not\kern0.2ex=}
% Text stuff
\def\AA{\text{\oldring{A}}}
\def\ZIF8{\mbox{ZIF-8}}
\def\ZIFCH3{\ZIF8(\ce{CH3})\xspace}
\def\ZIFCl{\ZIF8(Cl)\xspace}
\def\ZIFBr{\ZIF8(Br)\xspace}
\def\abinitio{\emph{ab initio}\xspace}
\def\Cudhbc{Cu(dhbc)$_\text{2}$(4,4$^\prime$-bpy)\xspace}
\def\RPMZn{RPM3-Zn\xspace}
\def\smallce#1{\ensuremath{\text{\tiny\ce{#1}}}}
\def\ie{\emph{i.e.}\xspace}
\def\etc{\emph{etc.}\xspace}
\def\vdW{van der Waals\xspace}
\def\cxx{{C\nolinebreak[4]\hspace{-.05em}\raisebox{.2ex}{\footnotesize ++}}\xspace}

% PSL cover page
\usepackage{psl-cover}
\pslassetspath{figures/psl}

\title{Simulation moleculaire multi-échelles de l'adsorption de fluides dans les matériaux nanoporeux flexibles}
\entitle{Molecular simulation of fluid adsorption in flexible nanoporous materials at multiple scales}
\author{Guillaume \textsc{Fraux}}

% !TEX root = thesis.tex
% Guillaume Fraux PhD thesis -- (C) 2019
% Distributed under CC-BY-SA-NC license 4.0
% Creative Commons Attribution-NonCommercial-ShareAlike 4.0 International

% information for cover pages

\institute{Chimie ParisTech}
\doctoralschool{Chimie Physique et\\ Chimie Analytique de\\ Paris Centre}{388}
\specialty{Chimie Physique}
\date{25 juin 2019}

\jurymember{1}{Caroline \textsc{Mellot-Draznieks}}{Directrice de Recherche, Collège de France}{Présidente}
\jurymember{2}{Sofía \textsc{Calero}}{Professeure, Universidad Pablo de Olavide}{Rapportrice}
\jurymember{3}{Paul \textsc{Fleurat-Lessard}}{Professeur, Université de Bourgogne}{Rapporteur}
\jurymember{4}{Renaud \textsc{Denoyel}}{Directeur de Recherche, Aix-Marseille Université}{Examinateur}
\jurymember{5}{Alain \textsc{Fuchs}}{Professeur, PSL Université}{Examinateur}
\jurymember{6}{François-Xavier \textsc{Coudert}}{Chargé de Recherche, Chimie ParisTech}{Directeur de thèse}

\frabstract{

Durant ma thèse, j'ai utilisé la simulation moléculaire pour étudier
l'adsorption et l'intrusion de fluides dans les matériaux nanoporeux flexibles.
Aujourd'hui, les matériaux à charpente organo-métallique appelés
\emph{metal-organic frameworks} (MOF) sont les principaux représentants de cette
famille de matériaux. Je me suis en particulier intéressé à la \ZIF8, un MOF
constitué de zinc et de ligands imidazolates organisés dans une topologie de
type sodalite. Grâce à la dynamique moléculaire quantique, j'ai pu montrer que
lors de l'adsorption de diazote dans la \ZIF8 il se produit une réorganisation
de la phase adsorbée qui augmente la quantité totale d'azote adsorbée. J'ai
aussi montré que changer la nature chimique des ligands permettait de supprimer
partiellement ou totalement cette réorganisation.

D'autre part, j'ai utilisé la dynamique moléculaire classique et les simulations
Monte-Carlo pour étudier l'adsorption et l'intrusion d'eau dans des matériaux
poreux hydrophobes. Ces matériaux ont des applications potentielles dans le
domaine du stockage et de la dissipation de l'énergie mécanique. La pression à
laquelle se produit l'intrusion, ainsi que la présence et la forme d'une boucle
d'hystérèse sont modifiable par l'ajout d'ions dans le liquide d'intrusion.
J'ai montré que liquide confiné dans la \ZIF8 ou dans des nanotubes
d'alumino-silicates appelés imogolites est fortement structuré, et que la
dynamique des molécules d'eau est ralentie par le confinement. La présence
d'ions modifie très peu la structuration, mais ralenti encore la dynamique, et
rigidifie l'ensemble du système. J'ai aussi étudié l'entrée d'ions dans la
\ZIF8, et observé une différence flagrante entre \ce{Li+} et \ce{Cl-} d'un
point de vue thermodynamique et cinétique.

Enfin, j'ai montré que la prise en compte de la flexibilité était nécessaire
pour prédire correctement la co-adsorption de gaz dans un matériau qui se
déforme (\emph{respiration, ouverture des fenêtres}, \etc) lors de l'adsorption.
Cette prise en compte est possible dans le cadre de la méthode \emph{Osmotic
Framework Adsorbed Solution Theory} (OFAST) pour décrire des changements de
phase du matériau.

}

\enabstract{

During my PhD, I used molecular simulation to study the adsorption and intrusion
of fluids in flexible nanoporous materials. As of today, metal-organic
frameworks (MOF) are the main example of this family of materials. I
specifically worked with \ZIF8, a MOF built with zinc metallic centers and
imidazolates linkers, organized in a sodalite topology. Using \abinitio
molecular dynamics I showed that nitrogen undergoes a reorganization inside the
pores during adsorption; increasing the total adsorbed amount. I also showed
that changes to the chemical nature of linkers allows to partially or completely
remove this reorganization.

On an other side, I used classical molecular dynamics and Monte Carlo
simulations to study adsorption and intrusion of water in hydrophobic porous
materials. These materials have possible applications in mechanical energy
storage and dissipation. The intrusion pressure, as well as the presence and
shape of an hysteresis loop, can be tuned by adding ions in the intrusion
liquid. I showed that the liquid confined in \ZIF8 or in alumino-silicate
nanotubes called imogolites is strongly structured; and that the water molecules
dynamic slows down under confinement. The presence of ions almost does not
modify the water structuration, but slows down dynamics even more, and makes the
whole system more rigid. I also studied ions entry in \ZIF8 structure, and
observed a clear difference between \ce{Li+} and \ce{Cl-} both on a
thermodynamic and kinetic point of view.

Finally, I showed that it is necessary to take in account flexibility to
correctly predict gas co-adsorption in frameworks that undergoes deformation
(\emph{breathing, gate-opening,} \etc) under adsorption. This is possible within
the scope of Osmotic Framework Adsorbed Solution Theory (OFAST) for materials
undergoing phase transition.

}

\frkeywords{simulation moléculaire, matériaux nanoporeux flexibles, intrusion, adsorption, simulations multi-échelles}
\enkeywords{molecular simulation, flexible porous materials, intrusion, adsorption, multi-scale simulations}


\addsectionbib{publications.bib}
\bibliography{thesis, publications}

\addto\captionsenglish{\renewcommand{\contentsname}{Table of contents}}

% DRAFT code, to be commented out/removed in final version
\makeatletter
    % Add a visual marker for overrfull boxes
    \overfullrule=2em
    % Output more data in log concerning underfull boxes
    % \showboxbreadth=50 \showboxdepth=50
    % Add the date/hour of compilation on each page
    \ct@draftingtrue

    % Use a big red box for TODO items
    \def\TODO{\colorbox{red}{TODO}}

    % Put a big red square for undefined references
    \def\@setref#1#2#3{%
        \ifx#1\relax
            \protect\G@refundefinedtrue
            \nfss@text{\reset@font\bfseries\colorbox{red}{??}}%
            \@latex@warning{Reference `#3' on page \thepage \space undefined}%
        \else
            \expandafter#2#1\null
        \fi%
    }
\makeatother

\begin{document}
    \maketitle

    \begingroup
        \hypersetup{hidelinks}
        \tableofcontents
    \endgroup

    % !TEX root = thesis.tex

\chapter*{General introduction}
\pagenumbering{arabic}

Nanoporous materials are material presenting nanometric cavities in their
structure. These cavities (called \emph{pores}) and the associated high specific
surface area are the foundation for crucial industrial applications, in
particular in the domains of fluid separation and storage, water purification,
and heterogeneous catalysis. Two classes of crystalline nanoporous materials
that particularly stand out today are zeolites and metal--organic frameworks.
Zeolites are natural and artificial porous aluminosilicate known since 1756 and
artificially synthesized since the 1940s. They are currently highly used at the
industrial level, in diverse applications including as catalysts in the oil
industry and water softeners in laundry detergents. Since the 2000s, a new
family of hybrid organic--inorganic crystalline nanoporous materials called
metal--organic frameworks have emerged and sparked interest in the scientific
community. These new materials are based on metallic clusters, linked together
by organic linkers. This composition offers them immense structural and chemical
diversity makes them tuneable: it is possible to engineer new materials with
specific pore sizes, shapes, and chemistry by using different combinations of
linkers and metals. The relatively weak coordination bonds between the metals
cations and organic linkers create intrinsic structural flexibility in
metal--organic frameworks, which can be either local or extended to the whole
material. Some of these metal--organic frameworks --- grouped together under the
name \emph{"soft porous crystal"} --- exhibit large scale structural
transformations under external stimuli such as temperature, pressure, adsorption
or even exposure to light.

Most if not all applications of nanoporous materials are related to the entry of
other chemical species (from the liquid or gas phase) inside the material's
pores. When the external fluid is in the gaseous state, the process is called
adsorption, while for liquids it is called intrusion. Both adsorption and
intrusion have an effect on the physical and chemical properties of the fluid
confined in the nanopores. Confined fluids are usually organized more regularly,
taking some aspect of a solid phase while remaining mobile. The opposite is also
true: the presence of the fluid inside the pores can modify the properties and
behavior of the surrounding material. Flexible materials are especially impacted
and can undergo adsorption-induced phase transitions, resulting in macroscopic
phenomena like gate opening, breathing or negative gas adsorption. The coupling
between adsorption or intrusion and changes in the structure of flexible
nanoporous materials is difficult to study, because it involves the equilibrium
between confined and bulk fluids, as well as equilibrium between different
phases of the material.

\newpage
I have been interested during my PhD in the molecular simulation of fluid
adsorption and intrusion in flexible nanoporous materials. Molecular simulation
tools can speed up the development of new materials tailored to specific
applications by predicting the properties of materials before they are
synthesized, lowering the cost of screening thousands of materials. Which
property can be studied, and the reliability of the corresponding prediction
depend on the techniques used to model the systems of interest. Over the course
of my PhD, I used multiple techniques at different time and length scales to
look at adsorption and intrusion in flexible porous materials, and the effects
on both the confined fluid and the materials. I used macroscopic models based on
classical thermodynamics for the study of co-adsorption of gases, umbrella
sampling simulations to extract free energy profiles during water intrusion,
classical molecular dynamics and Monte Carlo for the intrusion of water and
aqueous solutions in both rigid and flexible materials, and \abinitio molecular
dynamics simulations to investigate the molecular organization of gases inside
pores.

Unfortunately, current molecular simulations methods only allow looking at the
coupling between adsorption and deformations from a single point of view.
Molecular dynamics simulations can be used to describe deformations of a
supramolecular framework, but are unable to simulate open systems, in particular
the adsorption of particles in chemical equilibrium with a reservoir. Monte
Carlo simulation, on the other hand, can describe such open systems and thus the
adsorption phenomenon, but are very inefficient for the study of collective
deformations. I have thus taken interest in (and worked on) the Hybrid Monte
Carlo simulation method, which can give us the best of both Monte Carlo and
molecular dynamics simulations. I also approached molecular simulation in
general and Hybrid Monte Carlo in particular from the point of view of their
implementation in molecular simulation software, studying various simulations
techniques, their links to statistical thermodynamics and efficient
implementation strategy.

\begin{center}
    \pgfornament[width=6cm,color=CTsemi]{88}
\end{center}

This thesis presents my work on the molecular simulation of adsorption and
intrusion in flexible nanoporous materials, and is divided into six chapters. I
start by presenting nanoporous materials, in particular zeolites, metal--organic
frameworks and zeolitic imidazolate frameworks. I introduce the main
characteristics at the origin of their huge structural diversity. I also
describe the phenomena of adsorption and intrusion, how they relate to one
another, and what effects they can have both on nanoporous solids and on
confined fluids. In a second chapter, I review the classical thermodynamic
models of adsorption and co-adsorption, and then demonstrate how to use
classical thermodynamics with the OFAST theory to study co-adsorption in
flexible nanoporous materials. My third chapter establishes statistical
thermodynamics and the molecular simulations methods we can use to study
chemical systems: molecular dynamics and Metropolis Monte Carlo. I also briefly
discuss different techniques used to model the interactions between molecules,
such as \abinitio methods and classical force-fields. In chapter four, I present
the basis for density functional theory calculations, which I have used to study
the adsorption of nitrogen in \ZIF8, showing that the confined fluid undergoes a
reorganization linked to framework changes as the pressure increases. The fifth
chapter contains classical molecular dynamics studies of the adsorption and
intrusion of water in nanoporous solids. I studied intrusion of electrolyte
aqueous solutions in \ZIF8, and its impact on the fluid structure, dynamics, and
thermodynamics. I also looked at the confinement of water in aluminosilicate
nanotubes, and the corresponding impacts on structure and dynamics. The last
chapter describes my work on the implementation of molecular simulation
software, and in particular the Hybrid Monte Carlo simulation method, which is
particularly suited to the study of adsorption in flexible nanoporous materials.
Finally, I present some conclusion from my PhD project, and identify some
perspectives and challenges for future work.

\vfill
\begin{center}
    \pgfornament[width=6cm,color=CTsemi]{75}
\end{center}
\vfill\vfill

    \subfile{1-context}
    \subfile{2-macroscopic}
    \subfile{3-molsim}
    \subfile{4-abinitio}
    \subfile{5-classical}
    \subfile{6-software}
    % !TEX root = thesis.tex

\chapter*{General conclusions}

The work presented in this thesis is related to the study of adsorption and
intrusion in nanoporous flexible materials, the deformation of these materials
and the coupling between the two phenomena. Confining a fluid inside a porous
network has significant effects on its thermodynamics properties, due to the
competition between pore size and pore shape effects, and interface
interactions. This competition generates specific new behaviors, such as new
fluid phases and phases transitions, and is especially acute in nanoporous
materials, where the typical width of the pore and the range of the interactions
are of the same order of magnitude. On the other hand, the presence of a
confined fluid can also have strong effects on the surrounding solid, creating the
opportunity for new phases to exist and shifting the balance between multiple
meta-stable phases. This is particularly poignant in the case of \emph{flexible}
nanoporous material, such as many metal--organic frameworks.

Because these materials are relatively recent, their flexibility has often been
overlooked, and it was only in the recent years that the scientific community
started to take it into account. An example of such shift is presented in the
second chapter of this manuscript, with the incorporation of the osmotic
thermodynamic ensemble into the Ideal Adsorbed Solution Theory (IAST) for the
study of co-adsorption of gases, leading to the creation of the Osmotic
Framework Adsorbed Solution Theory (OFAST). In the aforementioned chapter, I
demonstrate that IAST is by construction invalid for the treatment of
co-adsorption when the adsorbing host is not inert during adsorption. In
particular, I show that IAST cannot be used for the prediction of co-adsorption
of fluid mixtures in frameworks presenting a gate-opening behavior, and that it
predicts non-physical selectivity, up to two orders of magnitude higher than
OFAST. Even when IAST is not explicitly used to compute selectivity in flexible
frameworks, researchers should be careful when comparing single-component
isotherms of different guests in presence of flexibility. Differences in step
pressure of stepped isotherms can lead to claims of strong selectivity, when
applying concepts that are valid only for rigid host matrices.

We should also take care of not going too far the other way either, and
attributing all observed behaviors to the flexibility of the material. In the third
chapter of this document, I used \abinitio molecular dynamics simulations to
explain the origin of a stepped isotherm for the adsorption of nitrogen in
\ZIFCH3 and \ZIFCl, and its absence in the similar framework of \ZIFBr. I showed
that while the framework does deform during adsorption for both \ZIFCH3 and
\ZIFCl, the deformations do not change the accessible volume and pore size
distribution of these materials. Instead, the increase in uptake in the isotherm
is linked to a reorganization of the fluid confined in the pores, reorganization
which does not happen in \ZIFBr because of the difference in pore size. It is
thus fundamental to account for both flexibility and confinement effects when
studying adsorption in soft porous crystals.

The same is true of intrusion, adsorption's big brother. In chapter five, I used
classical molecular simulations to study the confinement under high-pressure of
water and aqueous electrolyte solutions in \ZIF8, and imogolite nanotubes. I
observed confinement effects ranging from stronger spatial organization, changes
in elastic properties, to water dynamics slowdown. Interestingly, the presence
of ions at high concentrations can have the same effects on bulk water,
structuring the hydrogen bonds network and slowing down dynamics. The intrusion
of aqueous solution in hydrophobic material is a promising way to store and
dissipate mechanical energy. Adding electrolyte in the water at varying
concentration allows tuning the behavior and even switching from energy storage
to dissipation. I looked at the impact of the ions on the intrusion behavior
using umbrella sampling simulations to extract the free energy profile of entry
inside \ZIF8, showing that different ions have dissimilar barriers when
traversing \ZIF8 windows. This study is one of the first on the subject of
intrusion of electrolytes in metal--organic frameworks, and allowed to shed some
light on the complex behaviors emerging in these systems.

The need to simultaneously account for adsorption and deformations was a
recurrent theme of all these studies. But current simulations methods are only
able to address one dimension of the problem: molecular dynamics simulations can
describe deformations, but modeling open systems and thus adsorption is not
possible. Metropolis Monte Carlo simulations can be used for open systems, but
have difficulties efficiently sampling collective deformations. Hybrid Monte
Carlo is a possible answer to this dilemma, combining the efficiency of
molecular dynamics with the versatility of Monte Carlo simulations (in
particular the possibility of sampling open and extended ensembles). My last
chapter presents the hybrid Monte Carlo simulation method and how to use it for
direct simulations in the osmotic ensemble. Increasing the reach of such novel
methods within the scientific community requires them to be readily available in
generic, easy to use, and efficient software. I explored the design space for
such software with the Domino project, described in the last chapter of this
document.

There is another condition to fulfill before being able to widely use osmotic
ensemble simulations for the study of adsorption and intrusion in soft porous
crystals. We need to be able to predict the energy changes related to the
flexibility of the frameworks and their interactions with fluids. On one hand,
first-principle or \abinitio methods (such as the density functional theory)
enable to accurately compute the energy of arbitrary atomistic systems. On the
other hand, they require large amount of computational power, which prevents
them from being routinely used on large systems. When faced with such large
systems --- large in the number of atoms, the timescale of processes, or for
high-throughput screening --- we therefore often turn to classical force fields
instead.

Classical force fields are either accurate, \ie reproducing well the actual
potential energy surface; or transferable, \ie usable with multiple different
systems. Current transferable force fields are not well suited to describe the
flexibility arising from coordination bonds, so we need to create new force
fields for these systems. Historically, the parametrization of new force fields
has been a rather long and tedious process. In recent years, new machine
learning-based techniques for the consistent and fast derivation of accurate
force fields have been devised. I presented one of these techniques in the
fourth chapter of this document, and used it to derive force fields for \ZIF8
and some of its derivatives from \abinitio data. Such automated methods are
especially crucial for the study of metal--organic frameworks given the huge
diversity of their structures. I hope that the availability of both accurate
force fields and hybrid Monte Carlo simulations capable software will make it
easier to use molecular simulations to engineer new materials tailored for
specific applications.

Overall, this PhD presents multiple molecular modeling methods that can be used
for the study of adsorption and intrusion in flexible nanoporous materials. From
\abinitio simulations to remove the need for force field parametrization;
classical simulations using molecular dynamics to describe flexibility; Monte
Carlo and particularly Grand Canonical Monte Carlo for adsorption; free energy
methods such a umbrella sampling; to hybrid Monte Carlo for the direct
simulation of collective behaviors in open systems; all the way to macroscopic
modeling methods, based on classical thermodynamics.

\begin{center}
    \pgfornament[width=6cm,color=CTsemi]{88}
\end{center}

This work opens perspectives for improvements in various directions. Concerning
molecular simulation methods, hybrid Monte Carlo seems like a very powerful
technique, that can be used for a wide variety of systems outside of the problem
of adsorption in soft porous crystals. First, hybrid Monte Carlo, being a
Metropolis Monte Carlo method, will always converge to and sample the phase
space distribution of the correct statistical thermodynamic ensemble. This is in
opposition with molecular dynamics, which samples the micro-canonical ensemble
by default, and relies on thermostats and barostats to sample other ensembles.
These thermostats and especially barostats are not all equal, and only some
algorithms are able to precisely generate the correct ensemble. At the same
time, hybrid moves greatly improve the efficiency of Monte Carlo simulations by
taking into account the local curvature of the potential energy surface.

Whether the simulation of open systems is compatible with molecular dynamics is
still an open research question. On the opposite, such simulations are routinely
performed using Grand Canonical and Gibbs ensemble Monte Carlo. Hybrid Monte
Carlo could thus be used for simulations of open ensembles and dilute systems,
such as constant pH simulations, description of the ionic environment of
proteins, or the simulation of defects in crystalline materials, improving
efficiency compared to standard Monte Carlo, and giving access to such open
systems to molecular dynamics users. Thanks to the Metropolis Monte Carlo
acceptance scheme, there is no need for the short molecular dynamics run used
inside hybrid move to sample any actual thermodynamic ensemble or any physical
Hamiltonian. This property could be exploited to create even more efficient
Monte Carlo simulations, for example gradually inserting new molecules in a
system while simultaneously relaxing its environment. Compressible generalized
hybrid Monte Carlo is particularly promising, as it enables using a custom
Hamiltonian tailored for the problem at hand.

Another perspective for future work concerns classical force fields, and their
ability to accurately reproduce potential energy surfaces. The traditional
approach to force fields has been to decompose the energy as a sum of terms
depending on simple scalar values with physical meaning: distances, bond length,
bond angles, torsion dihedral angles, \etc. These scalar values are then
combined with simple mathematical expressions, such as power or exponential
functions. This approach prevents the potential energy surface from including
multi-body effects, and accurately reproducing the shape of \abinitio potential
energy surfaces used as references. Machine learning tools, in particular neural
network and Gaussian processes, can improve both areas. First, neural-networks
ability to reproduce arbitrary functions from $\mathds{R}^n$ to $\mathds{R}$ can
reduce the differences between the potential energy surfaces shapes. For
example, instead of imposing the Lennard-Jones functional form, neural networks
can reproduce the actual function. Secondly, machine learning algorithms can be
coupled with better descriptors of the atomic structure, accounting for
many-body effects. In recent years, multiple independent scientific teams worked
to design, train and evaluate such machine learning force fields and the
associated descriptors. To my knowledge, they remain to be used for the
simulation of nanoporous flexible materials.

\vfill
\begin{center}
    \pgfornament[width=6cm,color=CTsemi]{75}
\end{center}
\vfill\vfill


    \appendix
    \subfile{A-chemfiles}

    %!TEX root = A-web.tex
% Guillaume Fraux PhD thesis -- (C) 2019
% Distributed under CC-BY-SA-NC license 4.0
% Creative Commons Attribution-NonCommercial-ShareAlike 4.0 International

% List of publications
\clearpage
\begingroup
    % Format author name in bold: https://tex.stackexchange.com/a/73246
    \def\makenamesetup{%
        \def\bibnamedelima{~}%
        \def\bibnamedelimb{ }%
        \def\bibnamedelimc{ }%
        \def\bibnamedelimd{ }%
        \def\bibnamedelimi{ }%
        \def\bibinitperiod{.}%
        \def\bibinitdelim{~}%
        \def\bibinithyphendelim{.-}%
    }
    \newcommand*{\makename}[3]{\begingroup\makenamesetup\xdef#1{#2, #3}\endgroup}

    \newbibmacro*{name:bold}[2]{%
      \makename{\currname}{#1}{#2}%
      \makename{\findname}{\lastname}{\firstname}%
      \makename{\findinit}{\lastname}{\firstinit}%
      \ifboolexpr{ test {\ifdefequal{\currname}{\findname}}
                or test {\ifdefequal{\currname}{\findinit}} }{\itshape\bfseries}{}}

    \newcommand*{\boldname}[3]{%
        \def\lastname{#1}%
        \def\firstname{#2}%
        \def\firstinit{#3}%
    }

    \xpretobibmacro{name:family}{\begingroup\usebibmacro{name:bold}{#1}{#2}}{}{}
    \xpretobibmacro{name:given-family}{\begingroup\usebibmacro{name:bold}{#1}{#2}}{}{}
    \xpretobibmacro{name:family-given}{\begingroup\usebibmacro{name:bold}{#1}{#2}}{}{}
    \xpretobibmacro{name:delim}{\begingroup\normalfont}{}{}

    \xapptobibmacro{name:family}{\endgroup}{}{}
    \xapptobibmacro{name:given-family}{\endgroup}{}{}
    \xapptobibmacro{name:family-given}{\endgroup}{}{}
    \xapptobibmacro{name:delim}{\endgroup}{}{}

    \boldname{Fraux}{Guillaume}{G.}

    \begin{refsection}[publications.bib]
        \begin{refcontext}[sorting=ynt]
            \DeclareFieldFormat{labelnumberwidth}{#1}
            \nocite{*}

            \addtocontents{toc}{\protect\setcounter{tocdepth}{1}}
            \chapter*{List of Publications}
            \startcontents[chapters]
            \section*{Publications related to this work}
            % \printbibliography[heading=subsectionbibheading,title={Peer-reviewed Journals},type=article,env=publist,keyword={phd}]
            % \printbibliography[heading=subsectionbibheading,title={Preprints},type=unpublished,env=publist]
            \printbibliography[heading=none,type=article,env=publist,keyword={phd}]

            \section*{Publications linked to previous work}
            \printbibliography[heading=none,type=article,env=publist,keyword={previous}]
        \end{refcontext}
    \end{refsection}

    \boldname{}{}{}
\endgroup

% Reset TOC depth
\addtocontents{toc}{\protect\setcounter{tocdepth}{2}}


    \printbibliography

    \subfile{Z-resume-fr}

\end{document}
